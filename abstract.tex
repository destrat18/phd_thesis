Blockchains are a family of distributed consensus protocols that maintain a shared, append-only ledger without relying on a trusted central authority. Smart contracts are self-enforcing programs executed on blockchains. They enable applications across finance, healthcare, and supply chain management, and they secure substantial digital assets. This setting gives rise to natural optimization and verification problems, yet existing approaches often produce suboptimal results, lack strong formal guarantees, or do not scale. These limitations frequently compel developers to rely on manual audits, which are costly, error-prone, and can leave vulnerabilities undiscovered. This thesis develops theoretical and algorithmic foundations that bridge the gap between scalability and formal guarantees for smart contract analysis by capturing structural properties of contracts and their interactions through algebro-geometric methods and parameterized algorithms. These techniques advance prior work in applications including gas-cost analysis of contracts, maximizing miners' revenues, decentralized exchange routing, and compiler-level gas minimization. Beyond the core technical results, the thesis emphasizes that blockchain is inherently interdisciplinary and benefits from perspectives beyond computer science; cross-disciplinary methods can surface costly vulnerabilities that would otherwise remain undiscovered.
