Blockchains are a family of distributed consensus protocols that maintain a shared, append-only ledger without relying on a trusted central authority. Smart contracts are self-enforcing programs executed on blockchains. They support arbitrary complex logic. Their applications span multiple domains, including finance, healthcare, and supply chain management, in both the public and private sectors. They currently are in charge of billions of dollars' worth of digital assets.

Given their role, any flaw can have significant financial consequences. For example, the DAO attack of 2016 resulted in a theft of \$50,000,000. Moreover, in 2024 alone, Ethereum users spent more than \$2,350,000,000 on execution fees. Consequently, many natural optimization and verification problems arise in this domain, yet existing approaches often produce suboptimal results, lack strong assurances, or do not scale. These limitations frequently compel developers to rely on manual audits, which are costly, error-prone, and can leave vulnerabilities undiscovered.

This thesis develops algorithmic foundations that bridge the gap between scalability and reliable guarantees for smart contracts. We achieve this by capturing structural properties of contracts using techniques such as parameterized algorithms and algebro-geometric methods.
We demonstrate how these techniques can significantly advance the state-of-the-art solutions across domains such as gas estimation, compiler optimization, miner-revenue maximization, and decentralized-exchange routing.
Beyond the core technical results, the thesis emphasizes that blockchain is an inherently interdisciplinary field that benefits from perspectives beyond computer science. Cross-disciplinary methods can surface costly vulnerabilities that would otherwise remain undiscovered.

