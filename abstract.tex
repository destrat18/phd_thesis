Blockchains are a family of distributed consensus protocols that maintain a shared, append-only ledger without relying on a trusted central authority. Smart contracts are self-enforcing programs executed on blockchains. They support arbitrary complex logic. Their applications span multiple domains, including finance, healthcare, and supply chain management, in both the public and private sectors. They are currently in charge of billions of dollars' worth of digital assets. Given their role, any flaw can have significant financial consequences. Thus, many optimization and verification problems arise in this domain, yet existing approaches often produce suboptimal results, lack strong assurances, or do not scale. These limitations frequently compel developers to rely on manual audits, which are costly, error-prone, and can leave vulnerabilities undiscovered.

This thesis develops algorithmic foundations that bridge the gap between scalability and reliable guarantees for smart contracts. We achieve this by capturing structural properties of contracts using techniques from parameterized complexity theory, polyhedral geometry, and real algebraic geometry. We demonstrate how these techniques can significantly advance the state-of-the-art solutions across various domains, including gas estimation, compiler optimization, miner revenue maximization, and decentralized exchange routing. The thesis goes beyond core technical results, arguing that blockchain benefits from interdisciplinary perspectives that can uncover costly vulnerabilities otherwise overlooked.

