\chapter{Gas Upper-bounds for Smart Contracts}
\label{chp:asparagus}
This chapter originally appeard in the following publications:

\begin{itemize}
     \item Z. Cai, S. Farokhnia, A.K. Goharshady, and S. Hitarth, Asparagus: Automated Synthesis of Parametric Gas Upper-bounds for Smart Contracts, Proceedings of the ACM on Programming Languages, OOPSLA 2023
\end{itemize}


\newpage

\para{Background} In modern cryptocurrencies, a blockchain is a linked-list of blocks, which in turn contain a sequence of transactions. Anyone on the network can create and broadcast a transaction, but the transaction is only considered finalized when it is added to the blockchain. The blockchain is subject to consensus, i.e.~all nodes on the network will eventually agree on its contents. The consensus mechanism varies by cryptocurrency. For example, Bitcoin uses proof of work, whereas Ethereum runs on proof of stake. Irrespective of this, all modern cryptocurrencies heavily rely on miners. These are nodes who actively partake in running the consensus mechanism and extending the blockchain by adding new blocks. To incentivize the miners to add new transactions to the blockchain, each transaction includes a fee, which is paid to the miner who includes it in her block. This setting creates a natural optimization problem from the point-of-view of miners: Given a set of new transactions which are not yet added to the blockchain, how can we form an optimal block that maximizes the fees and thus the miner's revenue?

  		
	

\para{Our Contributions} This chapter provides algorithms and empirical results for optimal block production on Cardano and Ethereum:
