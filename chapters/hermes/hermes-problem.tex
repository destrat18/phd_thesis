\section{Exchange Rates as a Routing Problem}\label{sec:hermes-problem} 

\para{Decentralized Finance (DeFi)} Smart contracts have enabled the development of a new ecosystem that offers advanced, composable financial functions beyond basic token transfers on the blockchain, now collectively referred to as \emph{decentralized finance} (DeFi)~\cite{DBLP:conf/uss/McLaughlinKV23}.
DeFi has grown rapidly due to its transparent, trustless, and programmable financial services. Currently, the total value locked (TVL)\footnote{Total value locked refers to the aggregate value of all assets deposited in smart contracts.} in DeFi on Ethereum alone exceeds 65.9 billion USD~\cite{defillama, DBLP:journals/csur/XuPCF23}.\footnote{All prices in this paper are reported as of 15-06-2025.}

\para{Decentralized Exchanges (DEX)} Decentralized exchanges (DEXs) are a cornerstone of the DeFi ecosystem, enabling the permissionless trading of digital assets. Among the different DEX types, \emph{Automated Market Makers} (AMMs) have emerged as the most widely adopted by nearly every metric~\cite{zhang2024improved}. Central to AMMs is the concept of a \emph{liquidity pool}: a smart contract that holds reserves of two (or more) tokens, enabling users to trade between them without intermediaries. The exchange rate between two tokens within a liquidity pool is autonomously determined by a function of the current reserves. Uniswap is the leading DEX by TVL on Ethereum. Since 2024, Uniswap has hosted an average daily trading volume of $1.3$ billion USD, with an all-time cumulative trading volume reaching $2.26$ trillion USD~\cite{defillama}.

\para{Routing in Decentralized Exchanges} A fundamental challenge within the DEX ecosystem is discovering optimal exchange rates, a problem necessitated by the decentralized and fragmented liquidity landscape. Often, a direct trading pair between two assets does not exist, requiring a trader to perform a sequence of trades across several liquidity pools, a process known as \emph{routing}. 
Routing can be modeled as a graph problem on $G = (V, E, w)$, where the vertices $V$ represent tokens, the edges $E$ correspond to liquidity pools connecting pairs of tokens, and the weights $w$ encode the spot exchange rates available for trading between those tokens. A route is a sequence of tokens whose cumulative edge weights determine the final exchange rate. The goal is to find the route between two tokens that yields the best price. 
A well-known special case of routing is \emph{arbitrage}, where a trader earns risk-free profit by exploiting price discrepancies for the same asset across different pools. This problem is commonly modeled as finding a negative-weight cycle in the aforementioned graph~\cite{zhang2024improved,DBLP:conf/uss/McLaughlinKV23}. Arbitrages have been studied exhaustively in the literature on DEX due to their risk-free profits\cite{DBLP:journals/csur/XuPCF23}. However, the reported arbitrages constitute only a small fraction of trades, accounting for approximately $11.71\%$ of daily volume on Uniswap~\cite{theUniswapSubgraph,defillama}.


\para{Algorithmic Challenges in Routing}
While financial arbitrage has been extensively studied in both traditional and decentralized markets~\cite{DBLP:journals/csur/XuPCF23,zhang2024improved}, research focused on optimal routing within decentralized exchanges remains comparatively limited. The existing body of work reveals two primary algorithmic challenges. First, our study shows that current routing algorithms scale poorly as the number of assets increases. This issue is particularly acute given the operational constraints of modern blockchains, such as Ethereum's 12-second block time, and the dynamic nature of the ecosystem, which features continuously changing prices. Second, the presence of arbitrage opportunities, which manifest as negative-weight cycles in the graph-based problem formulation, renders standard shortest-path algorithms such as Bellman-Ford incapable of finding optimal routes. This situation has led to a dichotomy in existing solutions: they are either general-purpose algorithms that guarantee optimality but are often too slow for practical on-chain execution, or they rely on heuristics that are faster but offer no formal guarantees on the quality of the route.

\para{Our Focus}
In this chapter, we bridge this gap by designing a \textit{parameterized algorithm} tailored to the structural properties of DEX transaction graphs~\cite{Cygan2015}. The design of such algorithms follows a paradigm that diverges from traditional runtime analysis. Instead of measuring performance solely against the input size, a parameterized algorithm's efficiency is also analyzed with respect to an additional structural property of the input, known as a \textit{parameter}. For graphs, a common and powerful parameter is \emph{treewidth}, which measures how closely a graph resembles a tree. Our central thesis is that real-world DEX transaction graphs exhibit low treewidth. By exploiting this structural property, we have developed an algorithm that is both \textit{theoretically sound}, due to its formal complexity analysis, and \textit{practically efficient}, a claim we substantiate through extensive experiments on real-world DEX data. This approach strikes a crucial balance, achieving provable optimality without sacrificing the performance required in the demanding DEX environment.