\section{Prologue}

\para{Blockchain and Smart Contracts} This thesis focuses on designing automated algorithms for (i) optimization problems in blockchains and (ii) detecting vulnerabilities or mathematically verifying their absence in smart contracts. The recent success of Bitcoin~\cite{nakamoto2008bitcoin} has highlighted the potential of blockchains, encouraging researchers to extend the concept of transactions beyond the mere transfer of digital assets to the execution of arbitrary programs, known as smart contracts. Such architectures became prevalent because centralized systems are vulnerable to fraud, censorship, and manipulation, as they depend on a single authority for control and decision-making. Thus, smart contracts, which can employ any complex logic, remove the need for intermediaries by decentralizing control, making interactions more transparent, tamper-resistant, and censorship-proof. Consequently, they have become instrumental in holding digital assets valued at billions of dollars. Currently, Ethereum, the 2nd largest cryptocurrency with a market cap of $354,000,000,000$, and Cardano, the 9th largest with a market cap exceeding $14,000,000,000$, both support the execution of arbitrarily-complex smart contracts~\cite{coinmarketcap2024cryptocurrency}~\footnote{All prices and exchange rates are as of 01 January 2025.}. 


\para{Why Bother?} Vulnerabilities in smart contracts can have devastating consequences, making their security a critical concern. For example, the DAO attack in 2016 resulted in a loss of \$50,000,000, and a recent Bybit hack in February 2025 allowed attackers to steal more than \$1,500,000,000. These losses could have been prevented if the contracts had been formally and rigorously verified. In addition to security, there are also many unsolved or open-ended optimization problems in blockchains. The quintessential example is that users incur execution fees, known as ``gas'', when interacting with contracts. In 2024 alone, Ethereum users spent more than $\$2,700,000,000$ in gas fees. The Ethereum Foundation recognizes that these high fees present a substantial obstacle to broader adoption~\cite{farokhnia2023alleviating}. Thus, any vulnerability or inefficiency can have profound financial repercussions. Relying on unsound heuristics or suboptimal algorithms is unacceptable when billions of dollars are at stake. Rigorous, principled approaches are essential to ensure both the security and efficiency of blockchain systems.

\para{Real-World Applications of Smart Contracts} Smart contracts underpin a wide range of applications across both public and private sectors. In the public sector, they have been adopted in healthcare, crowdfunding, and supply chain management. For instance, the European Commission highlights properties such as trust, privacy, autonomy, inclusiveness, and transparency as key drivers for many projects launched to leverage decentralization for social and public good~\cite{polvora2020scanning}. Another notable example is the Bloxberg blockchain, led by the Max Planck Digital Library, which supports scientific research through smart contract infrastructure~\cite{librarynoyearfirst}. In the private sector, smart contracts have reshaped various financial services such as lending, insurance, and asset management, which are often referred to as decentralized finance (DeFi). Stablecoins like Tether (USDT) and USD Coin (USDC), with market capitalizations of $120,000,000,000$ and $50,000,000,000$, respectively, are implemented as Ethereum smart contracts~\cite{coinmarketcap2024cryptocurrency}. A further illustration is Uniswap, a decentralized exchange (DEX) that allows users to trade digital assets without intermediaries and currently secures $3,400,000,000$~\cite{labs2024uniswap}. Thus, the appeal of decentralization has drawn various professionals to blockchain. However, the cost of operations and potential vulnerabilities remain a significant concern for progress.

Given the rapid adoption of smart contracts, scalable and secure algorithms for verification and optimization are essential. Many formal methods provide soundness (and, where applicable, completeness) but do not scale, while heuristics can leave contracts vulnerable or inefficient. This thesis shows that, by exploiting structural properties of smart contracts, we can achieve the best of both worlds: principled methods that scale in practice. Building on prior work, we study two complementary approaches to these challenges.


\para{Parameterization} Many optimization and verification tasks can be framed as graph problems over CFGs of programs. These problems are often intractable. When dealing with intractable problems, e.g.~NP-hard graph problems, parameterized algorithms leverage the structural and sparsity properties of the underlying graphs. Unlike traditional complexity theory, which assesses runtime based solely on input size, parameterized complexity considers both the input size and an additional property of the instance (a ``parameter''). For instance, \textit{treewidth} is a well-known parameter that measures the tree-likeness of a graph, while \textit{treedepth} formalizes the star-like structure of a graph. Many problems that are computationally intractable (e.g.~NP-hard) on general graphs can be efficiently solved on graphs with small treewidth. The intuition is that, if a graph can be decomposed into small parts that are connected to each other in a tree-like manner, then one can treat the graph as if it were a tree and leverage tree-based algorithmic ideas such as bottom-up dynamic programming. It is well-known that CFGs of structured programs are sparse and tree-like, i.e.~they have bounded treewidth~\cite{thorup1998all}. The same has also been established for smart contracts~\cite{chatterjee2019treewidth}.

\para{Algebro-geometric methods}

\section{Outline}
% TODO: write at the end

\section{Summary of Contributions}
% TODO: write at the end


% \paragraph{Execution Costs.} There are also many unsolved or open-ended optimization problems in blockchains. The quintessential example is that users incur execution fees, known as ``gas'', when interacting with contracts. Smart contracts are executed by all users within the network, ensuring consensus on the final state of the contract. This poses a risk: a malicious user could invoke a function with an infinite loop or one that is too resource-hungry, blocking the entire network. To mitigate this risk, blockchains employ the concept of gas, assigning each low-level operation a predefined cost based on its computational complexity and resource consumption. The total execution cost, or transaction fee, is the cumulative gas used during execution. The users who invoke the contract must pay this fee. Furthermore, if execution reaches a user-defined or protocol-imposed gas limit (i.e., computation limit), the transaction is halted and raises an \textit{Out of Gas} exception. Even then, the user is still charged for the gas consumed up to that point since the computational resource has been used. While gas effectively mitigates the risk of denial-of-service attacks, it also results in high fees and limits the computational capacity of the blockchain network.


% \paragraph{Smart Contract Vulnerabilities.}
% Smart contracts are immutable, meaning they cannot be changed once deployed. Given their role in managing billions of euros in assets, any vulnerability can have significant financial consequences. For example, the DAO attack of 2016 resulted in a loss of $45,000,000$, and a recent hack in February 2025 allowed an attacker to steal more than $1,200,000,000$~\cite{FarokhniaG23sac,newsnoyearcryptocurrency}. Hence, smart contracts regularly undergo extensive security audits. However, they are usually developed in high-level languages and compiled to low-level bytecode, making compiler-level optimizations particularly sensitive. Any such optimization must ensure the correctness of smart contracts. Traditional testing is inadequate for safety-critical applications like these, as it only addresses a limited range of scenarios. Instead, we need formal verification and mathematical proofs of correctness for all possible cases. Indeed, the aforementioned attacks could have been prevented if the contracts had been formally and rigorously verified. Thus, the optimization process must be sound and secure, ensuring that the optimized contract preserves the semantics of the original.


\section{Awards}