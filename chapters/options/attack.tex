\section{Attack Cost} \label{sec:attack}

\subsection{Double-Spending Attack Success Rate}

Suppose $H$ is the total Bitcoin hashrate where $pH$ represents honest participants and $qH$ represents attackers' hash-rate, where $p+q = 1$. 

When a merchant's transaction is included in a block with a height of $B_0$, they must wait for $n$ blocks to ensure that the transaction has been successfully confirmed by honest participants since block height $B_0$. However, an attacker may be simultaneously mining a private chain with a height of $m$ from block $B_0$. 

This work~\cite{rosenfeld2014analysis} explains that to determine the probability of a successful double-spending attack, we can utilize a negative binomial variable. This model considers $m$ as the number of successes (blocks discovered by the attacker) before $n$ failures (blocks found by the honest network), with a probability of $q$ for success. Thus, the probability $r$ of a successful double-spend attack based on the confirmation block number n and q attacker's hash-rate is:
$$
r = 
\begin{cases}
    1-\sum_{m=0}^{n} \binom{m+n-1}{m}(p^n q^m - p^m q^n) & if\ q < p\\
    1 & if\ q >= p 
\end{cases}
$$

By using this formula, we can calculate the probability of a successful attack based on the confirmation block n and the attachker's hash-rate q in Table~\ref*{table:prob-succ}.

\begin{table}[H]
    % \centering
    \begin{tabular}{|c|c|c|c|c|c|c|c|c|}
\hline
\diagbox{q}{n}    & 1  & 2     & 3     & 4     & 5     & 6     & 7    & 8    \\ \hline
0.02 & 4  & 0.24  & 0.02  & 0     & 0     & 0     & 0    & 0    \\ \hline
0.04 & 8  & 0.93  & 0.12  & 0.02  & 0     & 0     & 0    & 0    \\ \hline
0.06 & 12 & 2.07  & 0.39  & 0.08  & 0.02  & 0     & 0    & 0    \\ \hline
0.08 & 16 & 3.64  & 0.91  & 0.24  & 0.06  & 0.02  & 0    & 0    \\ \hline
0.1  & 20 & 5.6   & 1.71  & 0.55  & 0.18  & 0.06  & 0.02 & 0.01 \\ \hline
0.12 & 24 & 7.95  & 2.86  & 1.07  & 0.41  & 0.16  & 0.06 & 0.03 \\ \hline
0.14 & 28 & 10.66 & 4.4   & 1.89  & 0.83  & 0.37  & 0.17 & 0.08 \\ \hline
0.16 & 32 & 13.72 & 6.35  & 3.05  & 1.5   & 0.75  & 0.38 & 0.19 \\ \hline
0.18 & 36 & 17.11 & 8.74  & 4.63  & 2.5   & 1.37  & 0.76 & 0.42 \\ \hline
0.2  & 40 & 20.8  & 11.58 & 6.67  & 3.92  & 2.33  & 1.4  & 0.85 \\ \hline
0.22 & 44 & 24.78 & 14.89 & 9.23  & 5.83  & 3.73  & 2.41 & 1.57 \\ \hline
0.24 & 48 & 29.03 & 18.65 & 12.34 & 8.31  & 5.66  & 3.9  & 2.7  \\ \hline
0.26 & 52 & 33.53 & 22.87 & 16.03 & 11.43 & 8.24  & 5.99 & 4.38 \\ \hline
0.28 & 56 & 38.26 & 27.53 & 20.32 & 15.23 & 11.54 & 8.81 & 6.77 \\ \hline
\end{tabular}

    \caption{Each cell shows the percentage of successful double-spending attack with confirmation blocks number $n$ and attachker's hash-rate $q$.}
    \label{table:prob-succ}
\end{table}

\todo{Check newer papers to get more accurate estimates}

\subsection{Hardware Cost}
For Bitcoin mining, ASICs are the superior option in terms of both speed and efficiency compared to GPUs. As a result, they are the practical choice for the majority of unknown2023bitcoin~\cite{\url{https:Asic}. There are a number of common ASICs listed in Table~\ref{table:asic} along with their TH/s (TeraHash) and their costs in US dollars.

\begin{table}[H]
    \centering
    \begin{tabular}{|c|c|c|}
    \hline
    Name                 & Hash Rate (TH/s) & Price (USD) \\ \hline
    Antminer S19 Pro     & 110               & 2860        \\ \hline
    WhatsMiner M30S++    & 112               & 3999        \\ \hline
    AVALONminer 1246     & 90                & 3890        \\ \hline
    WhatsMiner M32       & 68                & 3557        \\ \hline
    AvalonMiner 1166 Pro & 81                & 3000        \\ \hline
\end{tabular}

    \caption{Hash-rate and cost for 5 common ASICs.}
    \label{table:asic}
\end{table}
Through this paper we will use ``Antminer S19 Pro'' since it's among popular choice by unknown2023bitcoin~\cite{\url{https:10BestAsic}.

\todo{Update list of ASICs and Their prices}

\todo{Get estimation from marketplaces such as nicehash}

\subsection{Bitcoin Network Hash-Rate}
The Bitcoin hash rate $H$ estimates the number of hashes being generated by unknown2023bitcoin creating a block. Current rate is 443.38 EH/s (ExaHash).

\todo{average hash rate in a month :-?}

\subsection{Repeated Experiments}
By repeating the double-spend attack explained earlier, the chances of errors decrease and reliability improves. We can therefore model it as a simple Binomial Distribution, which considers the repeated trials based on success probabilities $r$ calculated in Table 1 and this formula:
$$
r' = 1 - (1-r)^n
$$

\subsection{Cost of Repeated Experiments}

For every value of $n$ and $q$, we calculate the probability $r'$ of at least one successful attack after $k$ repetitions. The cost of the attack $c_a$ depends on the number of unknownnoyearasic hardware needed to match $qH$, which is determined by the unknownnoyearasic hash-rate ($H_{asic}$) and $q$. To determine the cost, we use the following formula:
$$
C_a = \frac{qH}{H_{\text{asic}}}C_{\text{asic}}
$$

Which in this experiments $H_{asic}$ is 110 TH/s and $C_{\text{asic}}$ is $2860$ USD.

We rounded $r'$ to three decimal digits once we calculated the probability $r'$ for each $n$, $q$, and $k$. Our table~\ref{table:cost} provides the cost for the lowest number of repetitions $k$ required to achieve the highest probability for every $n$ and $q$.

\todo{Must include cost of Electricity based on $k$}

\begin{table}[H]
    \centering
    \section{Cost of the Attack} \label{sec:cost}

In the previous section, we showed that an attacker with a minority of the hash power can still succeed in reverting $6$ blocks with high probability. Of course, an attacker who has a majority of the hash power will succeed in reverting any number of blocks with probability $1.$ In this section, we consider the costs of a block reverting attack. Specifically, we ask the following question: \emph{How much does it cost to obtain a portion $q$ of the hash power?} Our goal is not to obtain an exact number, but a ballpark estimate and upper-bound on the cost. Thus, we make the following simplifying assumptions:
\begin{itemize}
	\item We only consider the cost of hardware at the time of writing. We assume the attacker is buying the hardware, rather than renting it and do not consider potential discounts on bulk orders.
	\item We ignore electricity costs as they vary widely based on location.
\end{itemize}
The justification for the first assumption is that it keeps our analysis sound, i.e.~we can only over-approximate the cost by making this assumption. As for the second assumption, we note that electricity costs are often negligible in comparison to hardware costs and that our main argument, i.e.~the vulnerability of Bitcoin to majority attacks and block-reverting attacks, remains intact even if the estimates we obtain here are doubled. Indeed, as we will soon see, the trade volume of Bitcoin derivatives is more than three orders of magnitude larger than the numbers obtained here.

At the time of writing, the total hash rate of the Bitcoin network is 566.12 EH/s\cite{\url{https:TotalHashRate}. 
Furthermore, \cite{\url{https:BitcoinAsicPrice} categorized Bitcoin mining ASICs into three efficiency tiers based on their energy consumption: Under 25 J/TH, 25-38 J/TH, and Above 38 J/TH. Figure~\ref{fig:asicPrice} shows the historical prices in USD per TH/s for each tier. To ensure the soundness of our analysis, we have estimated attack costs using the most expensive efficiency tier, i.e. Under 25 J/TH. This tier incorporates the most advanced generation of unknownnoyearasic hardware that consumes the least amount of electricity. Table~\ref{table:cost} summarizes the costs of obtaining various portions $q$ of the total hash power at the time of writing. Note that if the current hash power is $x,$ it is not enough for the attacker to purchase a hash power of $q \cdot x,$ since her hash power will also be added to the network's total. Instead, she should buy $y$ units of hash power where $\frac{y}{y+x}=q$ or equivalently $y = -\frac{q \cdot x}{q-1}.$ 
Moreover, Figure~\ref{fig:costHistorical} shows the historical price of obtaining various portions $q$ of the total hash power based on the data from~\cite{\url{https:BitcoinAsicPrice, \url{https:TotalHashRate, unknownnoyearcryptocurrency}. As Table~\ref{table:cost} and Figure~\ref{fig:costHistorical} show, it is easy to obtain a majority of the hash power, or a sizable minority that allows an attack as per the previous section, using an investment that is a tiny percentage of the Bitcoin's current market cap and, as we will see, three orders of magnitude smaller than the annual trade volume of Bitcoin derivatives.

\begin{figure}
    \centering
    \includegraphics[width=0.48\textwidth]{figures/asicPriceJTH.pdf}
    \caption{Historical prices for one TH/s of different Bitcoin mining ASICs grouped by four efficiency tiers.}
    \label{fig:asicPrice}
\end{figure}

% \tododone{add table. Note that when you buy $x$ units of hash, the total hash power of the network also increases by $x$. Take this into account. You should have $q = x / (x+ \text{current hash power of Bitcoin}).$ Add the table for $q = 0.1, 0.15, 0.2, \ldots, 0.5.$ The table should have the following columns: $q$, required TH/s, cost, (cost/current Bitcoin market cap)}

\begin{table}
	\centering
	\begin{tabular}{|c|c|c|c|}
    \hline
    q    & Required EH/s & Cost (Billion USD) & $\frac{\text{Cost}}{\text{Bitcoin Market Cap}}$ \\ \hline
    0.10 & 62.9023       & 0.7523             & 0.0005                                          \\ \hline
    0.15 & 99.9036       & 1.1948             & 0.0009                                          \\ \hline
    0.20 & 141.5301      & 1.6927             & 0.0012                                          \\ \hline
    0.25 & 188.7068      & 2.2569             & 0.0016                                          \\ \hline
    0.30 & 242.623       & 2.9018             & 0.0021                                          \\ \hline
    0.35 & 304.834       & 3.6458             & 0.0026                                          \\ \hline
    0.40 & 377.4136      & 4.5139             & 0.0032                                          \\ \hline
    0.45 & 463.1894      & 5.5397             & 0.0039                                          \\ \hline
    0.50 & 566.1204      & 6.7708             & 0.0048                                          \\ \hline
    \end{tabular}
	\caption{The required hash power and hardware cost for an attacker who wishes to control a portion $q$ of the total hash power.}
	\label{table:cost}
\end{table}




\begin{figure}
    \centering
    \includegraphics[width=0.48\textwidth]{figures/costHistorical.pdf}
    \caption{The historical cost of obtaining a portion $q$ of the hash power for various values of $q$.}
    \label{fig:costHistorical}
\end{figure}
    \caption{Cost table}
    \label{table:cost}
\end{table}

\begin{figure}[H]
    \centering
    \includegraphics[width=0.5\textwidth]{figures/poolHashrate.pdf}
    
    \caption{The attack threshold is the minimum amount of hashrate required to perform a mining pool attack, which is 18\% of the total hashrate. }
    \label{fig:poolHashrate}
\end{figure}
