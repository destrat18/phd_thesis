\section{Bitcoin Derivatives} \label{sec:options-future}
 
In the previous section, we have already established a rough upper-bound on the costs of block-reverting attacks and argued that this cost is less than one percent of the current market cap of Bitcoin. Nevertheless, the required investment is substantial. Thus, the only remaining piece of the puzzle is to argue why anyone would be incentivized to invest such huge sums with the hope of crashing Bitcoin's price. In other words, how can an attacker profit from a price crash if she is losing her mining rewards? 

\paragraph{Futures and Options (Derivatives)} A futures contract is a standard financial contract in which a seller agrees to sell something, in this case Bitcoin, to the buyer at a particular time and a preset price. For example, 1 Bitcoin is worth almost 70,000 USD at the time of writing. Alice and Bob can make a contract in which Alice promises to sell Bob 1 Bitcoin on 01-01-2025 (delivery date) at a price of 45,000 USD (delivery price). On the delivery date, Bob pays Alice 45,000 USD and Alice should either provide 1 Bitcoin or pay Bob the equivalent value in USD (actual price of Bitcoin on delivery date). In this case, we say that Alice is shorting Bitcoin, i.e. hoping that its value drops so that the 1 Bitcoin on 01-01-2025 is worth less than 45,000 USD. On the other hand, Bob is longing Bitcoin, i.e. hoping that its value increases or at least exceeds the delivery price of 45,000 USD. Usually, both sides have to provide a security deposit with a trusted third party to ensure they will honor their commitment. 

Futures are often traded as options, i.e. Alice can enter a contract with Bob which, in exchange for an upfront payment, gives Alice the option but not the obligation to sell 1 Bitcoin to Bob on or before 01-01-2025 at 45,000 USD. Alice can choose whether she would like to exercise the option or not. Some options can be exercised only on their delivery date, while others allow Alice to exercise her option at any time before 01-01-2025. Please see~\cite{soylemez2019cryptocurrency} for more details on cryptocurrency and especially Bitcoin derivatives.

These contracts, both futures and futures options, are in turn traded on future exchanges, i.e.~one can acquire Alice or Bob's interest in the contract in the same way as buying stocks. An attacker can acquire short positions in these contracts, i.e.~bet that the price of Bitcoin will fall, and then be in a position to achieve financial gains by intentionally crashing Bitcoin's value. 


\paragraph{Bitcoin Derivatives} There is a huge and mostly unregulated Bitcoin derivatives market whose trade volumes often exceed 750 billion or even 1 trillion USD in a calendar month. Cryptocurrency exchanges such as Binance, OKX and ByBit offer future contracts and options on a wide variety of coins. Figure~\ref{fig:options-UnregulatedTradeVolume} shows the monthly trade volume of these largely unregulated markets between August 2022 and March 2024~\cite{block2024crypto}. We note that there has been a recent explosion in the value of such derivative contracts, but their total value was already huge even before the recent uptick. In March 2024 alone, the volume was almost 2.5 \emph{trillion} USD. As evident in this figure, the annual trade volume of Bitcoin derivatives is an order of magnitude larger than Bitcoin's market cap, which was itself two orders of magnitude larger than the cost of an attack. Thus, it is clearly feasible to buy enough put options, or other shorting vehicles, to benefit from a price crash that is induced by the block-reverting attacks explained above.

\begin{figure}[H]
    \centering
    \includegraphics[width=\textwidth]{chapters/options/figures/UnregulatedTradeVolume.pdf}
    \caption{Total trade volume of unregulated Bitcoin derivatives~\cite{block2024crypto}.}
    \label{fig:options-UnregulatedTradeVolume}
\end{figure}


\paragraph{Open Interest of Bitcoin Options} At the time of writing, the open interest in BTC options is slightly more than 20 billion USD~\cite{block2024crypto}. Thus, a malicious party performing the attack mentioned in this chapter would need to obtain a considerable amount of the available put contracts. This may lead to market disruptions whose analysis is beyond the scope of this analysis. That said, if the derivatives market continues to grow and becomes much larger than it currently is, purchasing this amount of contracts might not even be detected.

