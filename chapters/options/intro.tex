
\section{Overview of the Attack}
\label{sec:options-intro}

\paragraph{Majority Attack~\cite{joshi2020survey, apontenovoa202151, shanaev2019cryptocurrency}} If an adversary controls more than half of the hash power, she can create arbitrarily deep forks and revert as many blocks as she wishes. This is often called a 51\% attack\footnote{This naming is erroneous since one does not actually need 51\% of the mining power but only more than 50\%. 
Thus, we prefer to use the term \emph{majority attack}.}. See Section~\ref{sec:prelim-bitcoin} for details on Bitcoin's architecture. To fork the blockchain from block $B_i$ onwards, she can simply create new alternative blocks $B'_{i+1}, B'_{i+2}, \ldots$ and continue mining on her own machine without disclosing the new alternative chain. She continues this until her alternative fork becomes longer than the network's current consensus chain, which is guaranteed to eventually happen with probability $1$ since she holds a majority of the hash power. At that point, the adversary can simply publish the new chain which will immediately become the consensus chain and thus replace all the previous blocks after $B_i$ and revert and potentially double-spend all the transactions therein. Not disclosing a valid block and instead continuing to mine to extend it is called \emph{selfish mining} and is widely studied in the literature on game-theoretic analysis of blockchain protocols~\cite{alarco2020selfish, goharshady2021irrationality, chatterjee2018quantitative}.  Specifically, it is now well-known that profiting from selfish mining is not limited to adversaries who have a majority of the hash power and can be done using a much smaller share~\cite{eyal2018majority, chatterjee2018ergodic}. The work~\cite{eyal2018majority} shows that an adversary that controls a minority stake (as a mining pool) can use selfish mining to increase its mining rewards and thus attract other miner to join it until it eventually controls a majority of the computational power. 

It is natural to wonder whether Bitcoin is actually vulnerable to majority attacks or any other attack that intentionally tries to revert multiple blocks. Given that the rule of thumb followed by most practitioners is to wait for 6 confirmations, a fork that goes 6 levels deep can very likely diminish the public's trust in Bitcoin and cause a crash in its market price. It is also widely accepted that a prolonged majority attack (if it happens) would be catastrophic to the cryptocurrency and can cause its downfall.  

\paragraph{Conventional Wisdom}  The conventional wisdom in the blockchain community is to assume that such block-reverting attacks are highly unlikely to happen. The reasoning goes as follows:
\begin{compactenum}
	\item Reverting multiple blocks and specifically double-spending a transaction that has 6 confirmations requires control of a majority of the mining power;
	\item Having a majority of the mining power is prohibitively expensive and requires an outlandish investment in hardware;
	\item Even if a miner, mining pool or group of pools does control a majority of the mining power, they have no incentive to act dishonestly and revert the blockchain, as that would crash the price of Bitcoin, which is ultimately not in their favor, since they rely on mining rewards denominated in BTC for their income. 
\end{compactenum}
There is some strong yet circumstantial evidence to support these, especially the latter claim. In 2014, the Ghash.io mining pool temporarily succeeded in breaching the 50\% threshold and controlling more than half of the hash power~\cite{hern2014bitcoin}. 
However, no attack was observed. Currently, the top two mining pools, i.e.~AntPool and Foundry USA, together control more than half of the computational power~\cite{blockchainComBitcoinHashrateDistribution}. 
Indeed, centralization of hash power has been an ongoing issue and the top 2-3 mining pools have often controlled more than half of the hash rate in the past months (Figure~\ref{fig:options-minerspool}), yet they have not attempted a majority attack. However, as we will outline in this chapter, all three assertions above are false and Bitcoin is indeed vulnerable to block reversion attacks. 

\begin{figure}
\includegraphics[width=0.9\linewidth]{chapters/options/figures/poolShare.pdf}
\caption{Market shares of the largest Bitcoin mining pools over time
~\cite{HashrateindexComBitcoinMiningPools}}
\label{fig:options-minerspool}
\end{figure}

\paragraph{Why 50\% is Not Necessary} As mentioned,~\cite{eyal2018majority} shows that selfish mining can become profitable and help an attacker reach a majority of the mining power even when the attacker begins with a much smaller hash rate. Notwithstanding the clever attack in~\cite{eyal2018majority}, an adversary who controls a portion $q$ of the total hash rate, even when $q$ is relatively small, can still attempt to perform selfish mining and create a fork that is several blocks deep. In such cases, the attacker's probability of success would be low and if she fails, she loses all the potential rewards she could have gained by honest mining. Nevertheless, if she does not care about such losses, e.g.~if she has a larger incentive to act dishonestly, then she can repeat the attack until she reverts 6 blocks. Specifically, as we will see, relying on an analysis similar to that of~\cite{rosenfeld2014analysis} shows that an attacker who has merely $30\%$ of the total hash rate has a success probability of more than $95\%$ if she performs the attack for 33.7 days. An attacker with $40\%$ of the hash rate can reach the same success probability within approximately 4.1 days and the time is reduced to only 1.9 days for an attacker with a $45\%$ share. 


\paragraph{The Costs of an Attack} The costs of an attack are also substantially smaller than one would expect. As we will see, a simple calculation shows that, at the time of writing, one can obtain the necessary hardware to have a majority of the hash power by spending approximately 6.77 billion USD. This is disregarding the potential discounts in bulk orders and assuming that the attacker buys the equipment rather than renting them. Crucially, although this number looks large, it is only 0.48 percent of the Bitcoin market cap at the time of writing and pales into insignificance in comparison with the trillion dollar monthly trade volume in Bitcoin derivatives. Similarly, we calculate that gaining a mining share of $20\%, 30\%$ and $40\%$ costs only 1.69, 2.90 and 4.51 billion USD respectively. Thus, putting this together with the results mentioned in the previous paragraph, a patient attacker who is willing to wait for longer can  dramatically reduce the costs of the attack\footnote{We are not considering electricity costs since they vary significantly in different countries. Nevertheless, we are intentionally grossly over-approximating the cost of the hardware, e.g.~by considering a purchase instead of renting. Moreover, the cost of electricity would not realistically surpass the hardware costs. Additionally, one can double all of our cost estimates and the analysis and vulnerabilities still stand.}.

\paragraph{Incentives for an Attack} Finally, and most importantly, the assumption that no attacker would have a financial incentive to perform such an attack is flatly wrong. By far the biggest threat is posed by the often-unregulated Bitcoin derivative contracts such as options and futures. As we will see, the monthly trade volume in Bitcoin derivatives was above 500 billion USD in 19 of the past 20 months and even reached a trillion USD in several calendar months, surpassing 2 trillion USD this past March. Thus, an attacker can first short Bitcoin and then have the incentive to intentionally crash its price. 

\paragraph{Summary of the Attack} In short, an attacker can first use the Bitcoin derivatives market to short Bitcoin by purchasing a sufficient amount of put options or other equivalent financial instruments. She can then invest any of the amounts calculated above, depending on the timeline of the attack, to obtain the necessary hardware and hash power to perform the attack. If the attacker chooses to obtain a majority of the hash power, her success is guaranteed and she can revert the blocks as deeply as she wishes. However, she also has the option of a smaller upfront investment in hardware in exchange for longer wait times to achieve a high probability of success. In any case, as long as her earnings from shorting Bitcoin and then causing an intentional price crash outweighs her investments in hardware, there is a clear financial incentive to perform such an attack. The numbers above show that the annual trade volume in Bitcoin derivatives is more than three orders of magnitude larger than the required investment in hardware. Thus, it is possible and profitable to perform such an attack. There is also a huge derivatives market on Bitcoin with trade volumes that often exceed 1 trillion USD per calendar month and have recently even exceeded 2 trillion USD per month (Section~\ref{sec:options-future}).

Based on the argument above, Bitcoin options and futures imperil Bitcoin's security and its core consensus protocol. We believe there is a complete lack of awareness on the part of financial players who are issuing such derivative contracts as to their potentially destructive effect on Bitcoin. We are hoping to raise awareness about this issue so that (i)~the financial institutions realize the risk that issuing Bitcoin derivatives, especially shorting vehicles such as put options, poses to their earnings, and (ii)~the regulators step up to reign in the unregulated derivatives market and enforce sensible caps on these trades. 

We also note that this vulnerability is, at its core, due to a disconnect between the financial players in the cryptocurrency markets on the one hand, who treat Bitcoin and other similar currencies as if they are publicly-traded stocks, and the decentralized design of proof-of-work on the other hand. If Bitcoin were a stock, the attack we are describing would be tantamount to an investor first shorting the stock using leveraged contracts for differences whose value far exceeds the company's market cap and then buying enough shares to take control of the company and intentionally crashing it. This would of course be bajpainoyearcountries due to a wide variety of insider trading regulations. However, since Bitcoin and its variants~\cite{chatterjee2019hybrid} use proof-of-work and do not follow a proof-of-stake protocol, they do not choose a miner randomly based on their stake as in~\cite{chen2019algorand, david2018ouroboros, cai2023game, abidha2024icbc, barakbayeva2024icbc, ballweg2023blockchain, fatemi2023secure, cai2023trustless, chatterjee2019probabilistic}. Instead, one only needs to control a large portion of the mining power, which is much cheaper. Moreover, mining is decentralized and largely unregulated and not subject to insider trading laws.

In the following sections, we will explain the calculations that went into each part of the argument above in more detail. 