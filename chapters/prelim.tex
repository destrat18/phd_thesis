\chapter{Preliminaries}
\label{chp:prelim}

\section{Blockchain} \label{sec:prelim-blockchain}
\paragraph{Blockchain} As pioneered by Bitcoin~\cite{nakamoto2008bitcoin}, most modern cryptocurrencies use a blockchain protocol. There are three fundamental objects: transactions, blocks, and the blockchain. Transactions are the basic units of record keeping, defining the cryptocurrency's history and order. In Bitcoin, transactions transfer money in the underlying cryptocurrency. Anyone can create and broadcast transactions, which must be validated, e.g.~by checking digital signatures to prove ownership. Valid transactions are spread across the network using a peer-to-peer gossip protocol. Valid transactions are grouped into blocks of fixed maximum size, and each block contains a hash pointer to the previous block, forming a singly-linked list called the blockchain. Each node maintains a local copy of the blockchain, so its history consists of all transactions in its chain, which also provides their ordering.

\paragraph{Smart Contracts} 
Bitcoin supports a limited scripting language for specifying conditions to spend a UTXO, such as requiring multiple signatures. In contrast, programmable blockchains, pioneered by Ethereum~\cite{wood2014ethereum}, allow arbitrary scripts in a Turing-complete language. The key idea is that blockchain consensus is independent of transaction type, enabling transactions beyond simple currency transfers. On Ethereum, a transaction can: (i)~transfer money, (ii)~deploy a \emph{smart contract}—a program in Ethereum Virtual Machine (EVM) bytecode, or (iii)~interact with existing smart contracts by calling their functions. A smart contract is a program added to the blockchain, making its code immutable. Each contract has dedicated storage and can hold currency. Once money is sent to a contract, it can only be recovered if the contract’s code transfers it elsewhere. Since the protocol provides consensus on transaction history, it also extends to the state of every contract, as all nodes can execute the transactions.




\paragraph{Transaction Fees} Given that consensus heavily relies on miners and that mining is often costly, especially in proof-of-work settings, the protocol must provide rewards to incentivize mining. The miners are paid a fixed reward for every block they add to the blockchain. This is also how new units of currency are created. Additionally, to incentivize the miners to create non-empty blocks, a user who creates a transaction can decide on a transaction fee, which will be paid to the miner who adds it to the blockchain. It is well-known in the community that transactions with low fees are often ignored by the miners.

\paragraph{Consensus mechanism} To ensure consensus and prevent double-spending, not every propagated transaction is finalized immediately; a consensus mechanism is required to make all nodes eventually agree on the blockchain's contents. Its purpose is to make adding new blocks subject to rules that prevent attackers from creating competing branches, or forks, which represent incompatible transaction histories. There are many consensus protocols~\cite{xu2023survey, dziembowski2015proofs, chatterjee2019hybrid, ball2017proofs, yin2019hotstuff}, with the most prominent being proof of work, used by Bitcoin~\cite{nakamoto2008bitcoin} and Ethereum Classic, and proof of stake~\cite{king2012ppcoin, kiayias2017ouroboros}, used by Ethereum~\cite{wood2014ethereum}, Cardano~\cite{david2018ouroboros}, and Algorand~\cite{chen2019algorand1}. In practice, different terms are used for miners; for example, Ethereum and most proof-of-stake currencies use \emph{validators} or \emph{block builders} to indicate they do not use proof of work. In this paper, we use the term \emph{miner} for all consensus mechanisms.

% TODO: Add definition for mining and miner
\paragraph{Mining} In proof of work, a node must solve a computational puzzle, such as partially inverting a hash function in Bitcoin, to add a new block. The chance of success is proportional to the computational power invested. This process is called \emph{mining}, and the nodes are called \emph{miners}. In proof-of-stake blockchains, a miner's chance of adding the next block is proportional to their stake, i.e., the number of coins they hold, and a random procedure selects a miner for each slot. The chosen miner decides the block's contents.

\paragraph{Transaction Fees} Given that consensus heavily relies on miners and that mining is often costly, especially in proof-of-work settings, the protocol must provide rewards to incentivize mining. The miners are paid a fixed reward for every block they add to the blockchain. This is also how new units of currency are created. Additionally, to incentivize the miners to create non-empty blocks, a user who creates a transaction can decide on a transaction fee, which will be paid to the miner who adds it to the blockchain. It is well-known in the community that transactions with low fees are often ignored by the miners.

% TODO add a general paragraph for UTXO and account-based


\section{Bitcoin}
\label{sec:prelim-bitcoin}

\paragraph{Bitcoin} Bitcoin~\cite{nakamoto2008bitcoin} was the first working protocol for a decentralized cryptocurrency and currently holds the largest market cap among all such currencies, 
amounting to more than 1.4 trillion USD at the time of writing~\cite{coinmarketcap2024cryptocurrency}\footnote{The time of writing is April 1st, 2024.}. 
There is also a huge derivatives market on Bitcoin with trade volumes that often exceed 1 trillion USD per calendar month and have recently even exceeded 2 trillion USD per month (Section~\ref{sec:options-future}).

\begin{figure}
	\center
	\includegraphics[width=0.8\linewidth]{chapters/options/bitcoin.pdf}
	\caption{A simplified view of the blockchain}
	\label{fig:options-bitcoin}
\end{figure} 

\paragraph{Proof-of-Work~\cite{nakamoto2008bitcoin, gervais2016security}} In Bitcoin, transactions are grouped into \emph{blocks} of a fixed maximum size. The blocks are then chained together in a singly-linked list using hash pointers with each block containing the hash of its parent (previous) block. This linked list is aptly named the \emph{blockchain}. The blockchain is subject to consensus, i.e.~all honest nodes on the network should eventually agree on its contents. Thus, adding a new block to the end of the blockchain is a deliberately hard task, called \emph{mining}, that requires the solution of a computationally-intensive hash inversion puzzle. This scheme is called \emph{proof-of-work} and ensures that a miner's chance of adding the next block to the blockchain is proportional to the miner's computational power, i.e.~how many hashes she can compute per unit of time.

Figure~\ref{fig:options-bitcoin} shows an overview of this process. In this figure we have omitted implementation details that are not relevant to this work. Each block $B_i$ contains the hash of the previous block $B_{i-1}.$ This serves as a hash pointer in the linked list. It also contains a nonce $n_i$ and a sequence of transactions $Tx_{i,1}, Tx_{i,2}, \ldots.$ A miner who aims to add a new block $B_{i+1}$ should first create the pointer to the previous block and populate a list of transactions that she intends to include. She should then choose the new nonce $n_{i+1}$ such that the hash $h(B_{i+1})$ of her new block is below a certain predefined threshold\footnote{In Bitcoin, the threshold changes dynamically to ensure that a new block is mined roughly every 10 minutes.}. Since the output of a hash function is unpredictable and an ideal cryptographic hash function can be modeled as a random oracle, the miner's only choice is to repeatedly try different nonces until she finds a valid block. Thus, her success probability is proportional to the number of hashes she can compute per unit of time.

Since mining is an expensive activity, due to both hardware and electricity costs, the miner should be financially incentivized to perform it. Bitcoin creates two incentives for the miner~\cite{meybodi2022optimal, nakamoto2008bitcoin, barakbayeva2024blockchain}: (a)~a block reward (currently 6.25 BTC $\approx$ 445,835 USD, expected to halve in almost three weeks from now) is paid to each miner who successfully adds a new block, and (b)~each transaction contains a transaction fee that is paid to the miner who adds it to the consensus chain.


\paragraph{Longest Chain Rule~\cite{nakamoto2008bitcoin, blum2020combinatorics}} In the  event that two miner find a valid block at approximately the same time, a temporary \emph{fork} happens in which there are two valid blockchains known to the network. In such a scenario, the Bitcoin protocol allows miner to try to extend either branch. However, as soon as a branch becomes longer than the other(s), the shorter branch(es) are dropped by everyone who honestly follows the protocol. Thus, the protocol mandates that the longest chain is always the consensus chain and that every node on the network must always consider the longest chain known to them as the authoritative blockchain\footnote{In practice, the length of a chain is not just the number of blocks in it, but rather the total difficulty of mining these blocks. However, this minor detail does not change any of the analyses in this work.}. This is illustrated in Figure~\ref{fig:options-longest}. As long as a majority of the computational power on the network follows the protocol honestly, all honest participants are guaranteed to eventually reach a consensus about the blockchain.

\begin{figure}
	\center
	\includegraphics[width=0.8\linewidth]{chapters/options/longest.pdf}
	\caption{An illustration of the longest chain rule in Bitcoin}
	\label{fig:options-longest}
\end{figure}

\paragraph{Double-Spending~\cite{rosenfeld2014analysis, nakamoto2008bitcoin, karame2012double, chaudhary2020double, karame2015misbehavior}} Preventing double-spending is arguably the main contribution of the Bitcoin protocol. A \emph{double-spending} attack is when a Bitcoin user tries to use the same coin (transaction output) in two different transactions. In such cases, the two transactions will be in conflict~\cite{meybodi2022optimal} and at most one of them can be added to the consensus chain. Specifically, if $Tx_1$ and $Tx_2$ both spend the same coin, then any proposed block that contains $Tx_1$ can only be valid if neither it nor any of its ancestors (previous blocks) contain $Tx_2.$
Suppose $\alice$ is selling an item to $\bob$ and $\bob$ is paying the price by a Bitcoin transaction $Tx_1$ that transfers part of his money to $\alice$. In this case, it is not enough for $\alice$ to see $Tx_1,$ since $\bob$ might have created a conflicting transaction $Tx_2$ that double-spends the same coin. Thus, $\alice$ should wait for $Tx_1$ to be added to the consensus chain. However, even this does not guarantee that the payment is finalized, since it is possible that the miner eventually create a longer chain that contains $Tx_2$ and thus consensus switches from $Tx_1$ to $Tx_2.$ In such cases, we would say that $Tx_1$ is \emph{reverted}. In practice, this is unlikely to happen if $Tx_1$ is already in a block $B_i$ and there are many blocks added after $B_i.$ Such blocks are called \emph{confirmation} blocks. The conventional wisdom and industrial standard practice is to wait for 6 confirmations before considering the transaction as irreversible, although some users take the risk of waiting for fewer confirmations~\cite{hou2020study}.
% TODO add a short sentence about majority attack

\section{Cardano}
\label{sec:prelim-cardano}

\paragraph{UTXO} In Bitcoin~\cite{nakamoto2008bitcoin}, each transaction has multiple inputs and outputs: inputs are coins entering the transaction, outputs are coins leaving. Each input must reference an output from a previous transaction, ensuring only previously received coins can be spent. To prevent double-spending, each output can be spent only once. Spendable coins are called Unspent Transaction Outputs (UTXOs). Although UTXOs can be reconstructed from history, Bitcoin nodes maintain a current UTXO set for efficient transaction and block validation. Since transactions cannot create new coins, the sum of outputs must be less than the sum of inputs; the difference is paid to the miner as a transaction fee. Thus, each transaction has a fixed, known fee independent of other transactions. This property also holds in other UTXO-based currencies like Cardano.


\section{Ethereum}
\label{sec:prelim-ethereum}


\paragraph{Account Model} There is no concept of UTXO on Ethereum. Instead, every node on the Ethereum network keeps track of the so-called \textit{world state}~\cite{wood2014ethereum}. The world state consists of a number of \textit{accounts}. Each account has a balance, i.e.~the number of coins it possesses denominated in Wei. A Wei is $10^{-18}$ Ether. An account can either be owned by an external party, e.g.~a human user, or correspond to a smart contract. In the latter case, the account state also includes the contract's code and all data stored in the smart contract, e.g.~values of global variables. All transactions must be initiated by externally-owned accounts. In other words, contracts do not run automatically. They are only executed when their functions are invoked by an external account or another contract.


\paragraph{Gas} As the EVM language is Turing-complete, one can write and invoke contracts whose execution uses huge or even infinite time and memory. This is undesirable given that every transaction has to be run by all nodes on the network. Thus, intentionally executing long or resource-intensive contracts is a type of denial-of-service (DoS) attack. To combat this, Ethereum introduced the concept of \emph{gas}~\cite{foundation2025gas}. Gas is a measure of the total resources required to run an invocation. A gas cost is assigned to every atomic operation in the EVM language, including memory usage or deploying code. The costs of atomic operations are part of the protocol and provided as a table in~\cite{dameron2018beigepaper}. The initiator of a transaction has to pay a transaction fee proportional to the total gas cost of its execution. This effectively disincentivizes long and resource-hungry executions. Moreover, every block has a gas limit, which was originally set at 30,000,000 units of gas but has recently increased to 36,000,000. If the sum of gas usages of all transactions in a block exceeds this limit, the block is considered invalid and will not be added to the blockchain. This ensures that every node will have to perform only a limited amount of computation for each block.