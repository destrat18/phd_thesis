\chapter{Preliminaries}
\label{chp:prelim}

\section{Blockchain} \label{sec:prelim-blockchain}
\paragraph{Blockchain} As pioneered by Bitcoin~\cite{bitcoin}, most modern cryptocurrencies use a blockchain protocol. There are three fundamental objects: transactions, blocks, and the blockchain. Transactions are the basic units of record keeping, defining the cryptocurrency's history and order. In Bitcoin, transactions transfer money in the underlying cryptocurrency. Anyone can create and broadcast transactions, which must be validated, e.g.~by checking digital signatures to prove ownership. Valid transactions are spread across the network using a peer-to-peer gossip protocol. Valid transactions are grouped into blocks of fixed maximum size, and each block contains a hash pointer to the previous block, forming a singly-linked list called the blockchain. Each node maintains a local copy of the blockchain, so its history consists of all transactions in its chain, which also provides their ordering.

\paragraph{Smart Contracts} 
Bitcoin supports a limited scripting language for specifying conditions to spend a UTXO, such as requiring multiple signatures. In contrast, programmable blockchains, pioneered by Ethereum~\cite{wood2014ethereum}, allow arbitrary scripts in a Turing-complete language. The key idea is that blockchain consensus is independent of transaction type, enabling transactions beyond simple currency transfers. On Ethereum, a transaction can: (i)~transfer money, (ii)~deploy a \emph{smart contract}—a program in Ethereum Virtual Machine (EVM) bytecode, or (iii)~interact with existing smart contracts by calling their functions. A smart contract is a program added to the blockchain, making its code immutable. Each contract has dedicated storage and can hold currency. Once money is sent to a contract, it can only be recovered if the contract’s code transfers it elsewhere. Since the protocol provides consensus on transaction history, it also extends to the state of every contract, as all nodes can execute the transactions.




\paragraph{Transaction Fees} Given that consensus heavily relies on miners and that mining is often costly, especially in proof-of-work settings, the protocol must provide rewards to incentivize mining. The miners are paid a fixed reward for every block they add to the blockchain. This is also how new units of currency are created. Additionally, to incentivize the miners to create non-empty blocks, a user who creates a transaction can decide on a transaction fee, which will be paid to the miner who adds it to the blockchain. It is well-known in the community that transactions with low fees are often ignored by the miners.

\paragraph{Consensus mechanism} To ensure consensus and prevent double-spending, not every propagated transaction is finalized immediately; a consensus mechanism is required to make all nodes eventually agree on the blockchain's contents. Its purpose is to make adding new blocks subject to rules that prevent attackers from creating competing branches, or forks, which represent incompatible transaction histories. There are many consensus protocols~\cite{xu2023survey,DBLP:conf/crypto/DziembowskiFKP15,DBLP:conf/sac/ChatterjeeGP19,DBLP:journals/iacr/BallRSV17a,DBLP:conf/podc/YinMRGA19}, with the most prominent being proof of work, used by Bitcoin~\cite{bitcoin} and Ethereum Classic, and proof of stake~\cite{king2012ppcoin,DBLP:conf/crypto/KiayiasRDO17}, used by Ethereum~\cite{wood2014ethereum}, Cardano~\cite{DBLP:conf/eurocrypt/DavidGKR18}, and Algorand~\cite{DBLP:journals/tcs/ChenM19}. In practice, different terms are used for miners; for example, Ethereum and most proof-of-stake currencies use \emph{validators} or \emph{block builders} to indicate they do not use proof of work. In this paper, we use the term \emph{miner} for all consensus mechanisms.

% TODO: Add definition for mining and miner
\paragraph{Mining} In proof of work, a node must solve a computational puzzle, such as partially inverting a hash function in Bitcoin, to add a new block. The chance of success is proportional to the computational power invested. This process is called \emph{mining}, and the nodes are called \emph{miners}. In proof-of-stake blockchains, a miner's chance of adding the next block is proportional to their stake, i.e., the number of coins they hold, and a random procedure selects a miner for each slot. The chosen miner decides the block's contents.

\paragraph{Transaction Fees} Given that consensus heavily relies on miners and that mining is often costly, especially in proof-of-work settings, the protocol must provide rewards to incentivize mining. The miners are paid a fixed reward for every block they add to the blockchain. This is also how new units of currency are created. Additionally, to incentivize the miners to create non-empty blocks, a user who creates a transaction can decide on a transaction fee, which will be paid to the miner who adds it to the blockchain. It is well-known in the community that transactions with low fees are often ignored by the miners.

\paragraph{Transaction Fees} Given that consensus heavily relies on miners and that mining is often costly, especially in proof-of-work settings, the protocol must provide rewards to incentivize mining. The miners are paid a fixed reward for every block they add to the blockchain. This is also how new units of currency are created. Additionally, to incentivize the miners to create non-empty blocks, a user who creates a transaction can decide on a transaction fee, which will be paid to the miner who adds it to the blockchain. It is well-known in the community that transactions with low fees are often ignored by the miners.

% TODO add a general paragraph for UTXO and account-based


\section{Cardano Architectures}
\label{sec:prelim-cardano}

\paragraph{UTXO} In Bitcoin~\cite{bitcoin}, each transaction has multiple inputs and outputs: inputs are coins entering the transaction, outputs are coins leaving. Each input must reference an output from a previous transaction, ensuring only previously received coins can be spent. To prevent double-spending, each output can be spent only once. Spendable coins are called Unspent Transaction Outputs (UTXOs). Although UTXOs can be reconstructed from history, Bitcoin nodes maintain a current UTXO set for efficient transaction and block validation. Since transactions cannot create new coins, the sum of outputs must be less than the sum of inputs; the difference is paid to the miner as a transaction fee. Thus, each transaction has a fixed, known fee independent of other transactions. This property also holds in other UTXO-based currencies like Cardano.


\section{Ethereum Architectures}
\label{sec:prelim-ethereum}


\paragraph{Account Model} There is no concept of UTXO on Ethereum. Instead, every node on the Ethereum network keeps track of the so-called \textit{world state}~\cite{wood2014ethereum}. The world state consists of a number of \textit{accounts}. Each account has a balance, i.e.~the number of coins it possesses denominated in Wei. A Wei is $10^{-18}$ Ether. An account can either be owned by an external party, e.g.~a human user, or correspond to a smart contract. In the latter case, the account state also includes the contract's code and all data stored in the smart contract, e.g.~values of global variables. All transactions must be initiated by externally-owned accounts. In other words, contracts do not run automatically. They are only executed when their functions are invoked by an external account or another contract.


\paragraph{Gas} As the EVM language is Turing-complete, one can write and invoke contracts whose execution uses huge or even infinite time and memory. This is undesirable given that every transaction has to be run by all nodes on the network. Thus, intentionally executing long or resource-intensive contracts is a type of denial-of-service (DoS) attack. To combat this, Ethereum introduced the concept of \emph{gas}~\cite{ethereumgasfees}. Gas is a measure of the total resources required to run an invocation. A gas cost is assigned to every atomic operation in the EVM language, including memory usage or deploying code. The costs of atomic operations are part of the protocol and provided as a table in~\cite{dameron2018beigepaper}. The initiator of a transaction has to pay a transaction fee proportional to the total gas cost of its execution. This effectively disincentivizes long and resource-hungry executions. Moreover, every block has a gas limit, which was originally set at 30,000,000 units of gas but has recently increased to 36,000,000. If the sum of gas usages of all transactions in a block exceeds this limit, the block is considered invalid and will not be added to the blockchain. This ensures that every node will have to perform only a limited amount of computation for each block.