% For better font rendering
% \usepackage{lmodern}
% For citations and references (adjust style as needed)
%HS: Add following for fullcite
%\usepackage[backend=biber, style=authoryear]{biblatex}
%\addbibresource{refs.bib} % Replace with your .bib file name
%\usepackage{natbib}
% make references clickable 
% \usepackage{unicode-math}
% \setmathfont{Latin Modern Math} % or another math font if you prefer

\usepackage[cal=cm,bb=ams,scr=boondoxo]{mathalfa} % One family for various math families
\usepackage{bibentry}
\usepackage[hidelinks, pdfpagelabels]{hyperref}
\usepackage{makecell}
\usepackage{rotating}
\usepackage{mathtools}
% \usepackage{amsfonts}
\usepackage{nicefrac}
\usepackage{float}
\usepackage{wrapfig}
\usepackage{amscd}
\usepackage{amsthm}
\usepackage{verbatim}
\usepackage{tikz}
\usepackage{pgfplots}
\pgfplotsset{compat=1.18}

\usetikzlibrary{calc, matrix, shapes, arrows, positioning, decorations.pathreplacing, backgrounds}
\usetikzlibrary{shapes.geometric, fit}
\usetikzlibrary{arrows.meta}
\usepackage[many]{tcolorbox}
\usepackage{color,colortbl}
\usepackage{adjustbox}
\usepackage{changepage}
\usepackage{framed}
\tikzstyle{box_nofill}=[rectangle,draw]
\tikzstyle{box_fill1}=[rectangle,draw,fill=asparagus!50,minimum size=1.4em]
\tikzstyle{box_fill2}=[rectangle,draw,fill=babyblueeyes!50,minimum size=1.4em]
% \usepackage{subcaption}
% \usepackage{pgfplots}
% \usepackage{thmtools}

\usepackage{algorithm}
\usepackage[noend]{algpseudocode}

\usepackage[utf8]{inputenc}
\usepackage[T1]{fontenc}
\usepackage{xspace}

\usepackage{amssymb}
\usepackage{stmaryrd}
% The stmaryrd font family doesn't provide a bold series; declare a
% substitution so LaTeX won't warn when a bold series is requested.
% If the stmaryrd FD file is available, load it first and then declare
% the bold->medium substitution for the stmry family. This avoids the
% "font family unknown" error when the FD file hasn't been read yet.
% Define the stmry math symbol font and map its bold version to the
% medium series so that bold math (e.g. via \bm) won't request a
% non-existent bold font.
\DeclareSymbolFont{stmry}{U}{stmry}{m}{n}
\SetSymbolFont{stmry}{bold}{U}{stmry}{m}{n}

% Suppress Overfull/Underfull \hbox and \vbox warnings (reduces noisy
% diagnostics in editors like VS Code). Adjust if you want looser
% control instead of fully silencing these messages.
% - \hfuzz: ignore overfull boxes smaller than this glue (in pt)
% - \hbadness: threshold for reporting underfull hbox (0..10000)
% - \vfuzz/\vbadness analogues for vertical boxes
% - \overfullrule: width of visible box in output; 0pt hides it
\hfuzz=100pt
\hbadness=10000
\vfuzz=100pt
\vbadness=10000
\overfullrule=0pt
\usepackage{etoolbox}
\usepackage{bm}
% \usepackage{cite}
%HS: Removing cite for hkust thesis template
\usepackage{textcomp}
\usepackage{tabto}
\usepackage{thm-restate}
\usepackage{blkarray}
\usepackage{refcount}

% \usepackage{lineno}
% \linenumbers

\usepackage{caption}

\usepackage{xcolor}
\def\BibTeX{{\rm B\kern-.05em{\sc i\kern-.025em b}\kern-.08em
    T\kern-.1667em\lower.7ex\hbox{E}\kern-.125emX}}


\usepackage{svg}

% \usepackage{algpseudocode}
% \usepackage[ruled,linesnumbered]{algorithm2e}

\definecolor{gray}{rgb}{0.3, 0.3, 0.3}

% \SetCommentSty{\textcolor{gray}{\#}} % Gray color for comments
% \SetKwComment{tcp}{\color{gray}$\blacktriangleright$ }{}

% % Redefine comment style
% \SetCommentSty{textit} % Italicize comment
% \SetCommentSty{color{gray}} % Set color to gray
% \SetKwComment{Comment}{$\blacktriangleright$\ }{} % Use black triangle



% \usepackage{setspace}
\usepackage{listings}
\usepackage{pgf}
\usepackage{graphicx}
% \usepackage{subfig}
\usepackage{dashbox}
\usepackage{float}
\usepackage{syntax}
% \usepackage{hyperref}
\usepackage{hhline, booktabs}
\usepackage{array,multirow}	
\usepackage{tablefootnote}
% \usepackage{subcaption}
% \usepackage{longtable} % For long tables
\usepackage{cleveref}
% \usepackage{mathrsfs}
\usepackage{subcaption} 




\setlength{\fboxsep}{1pt}
\setlength{\dashlength}{4pt}
\newcommand{\codetemplate}[1]{\dbox{$#1$}}
\newcommand{\Hole}[1]{\dbox{$#1$}}
\newcommand{\errorRejected}{\textit{Failed}}
\newcommand{\errorRose}{\textit{Failed}}
\newcommand{\errorUnsat}{\textit{Unsat}}
\newcommand{\errorTimeout}{\textit{Timed out}}
\usepackage{paralist}
\usepackage{lscape}

\newcommand{\hit}[1]{\textcolor{red}{\textbf{Hitarth: }\textit{#1}}}

\usepackage{amsmath}
% \usepackage{thmtools}

\usepackage{thmtools}
\usepackage{thm-restate}


\theoremstyle{plain}
% \newtheorem{example}{Example}
% \newtheorem{definition}{Definition}
\newtheorem{namedtheorem}{Theorem}
\newtheorem{namedlemma}{Lemma}

\newtheoremstyle{noparens}% cf. p. 10 of user guide of 'amsthm' package
    {}{}{\itshape}{}%
    {\bfseries}{.}{ }%
    {\thmname{#1}\thmnumber{ #2}\thmnote{ {(\mdseries #3)}}}

\theoremstyle{noparens}
\newtheorem{theorem}[namedtheorem]{Theorem}
\newtheorem{block2024crypto}{Open problem}
\newtheorem{proposition}{Proposition}
\newtheorem{conjecture}{Conjecture}
\newtheorem*{fact*}{Fact}
\newtheorem{excont}{Example}
\newtheorem{corollary}{Corollary}
\newtheorem{lemma}{Lemma}
\newtheorem{claim}{Claim}
\newtheorem{definition}{Definition}
\newtheorem{example}{Example}
\theoremstyle{remark}
\newtheorem{remark}{Remark}
\newtheorem*{remark*}{Remark}


\renewcommand{\theexcont}{\theexample}
% \let\example\relax
% \declaretheorem[style=remark]{Example}

% \renewcommand{\paragraph}[1]{\smallskip\noindent\textbf{\emph{#1}}} % Redefine paragraph to be bold and italicized

% \renewcommand{\paragraph}[1]{\smallskip\noindent\textbf{\emph{#1}}} % Redefine paragraph to be bold and italicized

\lstset
{ %Formatting for code in appendix
    numbers=left,
    stepnumber=1,
    showstringspaces=false,
    tabsize=1,
    breaklines=true,
    breakatwhitespace=false,
}


\usepackage{listings}
\definecolor{codegreen}{rgb}{0,0.6,0}
\definecolor{codegray}{rgb}{0.5,0.5,0.5}
\definecolor{codepurple}{rgb}{0.58,0,0.82}
\definecolor{backcolour}{rgb}{0.95,0.95,0.92}
\definecolor{gruen}{rgb}{0.0, 0.5, 0.0}
\definecolor{rot}{rgb}{1.0, 0.13, 0.32}

% \lstdefinestyle{C}{
% morekeywords={then}
% }

% \lstset{
%     keywords={then},
%     keywordstyle=\color{red}
% }    

\lstset{
basicstyle=\ttfamily,columns=flexible,frame=single,framerule=0pt,%
	%backgroundcolor=\color{gray!20},%
	xleftmargin=\fboxsep,%
	xrightmargin=\fboxsep,
	language=[LaTeX]TeX,%
	numbers=left,
	keywordstyle=\color{blue},%
	texcsstyle=*\color{red}\bfseries,%
	texcs={end,begin,documentclass,graphicspath},%
	mathescape=false,escapechar=|,%
	literate={<B>}{\textcolor{blue}{\string\usepackage}}1
	{\{ }{\textcolor{blue}{\{}}1
	{\}}{\textcolor{blue}{\}}}1
	{[}{\textcolor{blue}{[}}1     
	{]}{\textcolor{blue}{]}}1
        {then}{\textcolor{blue}{then }}1
}
\lstset{emph={%  
		@prog, @real, @pre, @post, @invariant, @macro, @var, @recurrence, %
	},emphstyle={\color{red}\bfseries}%
}%

% \usepackage{xcolor}


\usepackage{enumitem}
\newlist{mycompactitem}{itemize}{3} % 3 is max-depth
\setlist[mycompactitem]{leftmargin=0em,label=\textbullet, nosep}
\newlist{mycompactenum}{enumerate}{3} % 3 is max-depth
\setlist[mycompactenum]{leftmargin=.6em,label=\textnormal{(\arabic*)},nosep}

\newcommand{\aoverb}[2]{$\substack{\text{#1} \\ \text{#2}}$}


% \lstset{basicstyle=\footnotesize\ttfamily,columns=flexible,frame=single,framerule=0pt,%
% 	%backgroundcolor=\color{gray!20},%
% 	xleftmargin=\fboxsep,%
% 	xrightmargin=\fboxsep,
% 	language=[LaTeX]TeX,%
% 	numbers=left,
% 	keywordstyle=\color{blue},%
% 	texcsstyle=*\color{red}\bfseries,%
% 	texcs={end,begin,documentclass,graphicspath},%
% 	mathescape=false,escapechar=|,%
% 	literate={<B>}{\textcolor{blue}{\string\usepackage}}1
% 	{\{ }{\textcolor{blue}{\{}}1
% 	{\}}{\textcolor{blue}{\}}}1
% 	{[}{\textcolor{blue}{[}}1     
% 	{]}{\textcolor{blue}{]}}1
% }
% \lstset{emph={%  
% 		@prog, @real, @pre, @post, @invariant, @macro%
% 	},emphstyle={\color{red}\bfseries}%
% }%


\usepackage{pdfrender}
\newcommand*{\boldcheckmark}{%
	\textpdfrender{
		TextRenderingMode=FillStroke,
		LineWidth=1pt, % half of the line width is outside the normal glyph
	}{\checkmark}%
}




\lstdefinestyle{smtlib-style}
{
  language=Lisp,
  frame=single,
  % basicstyle=\ttfamily,
  keywordstyle = [1]{\color{Gray}},
  keywordstyle = [2]{\color{Blue}},
  otherkeywords = {;,<<,>>,++},
  morekeywords = [1]{;},
  keywords = [2]{define-fun, define-literal, define-let, define, declare-sort, declare-fun, LINE, (, )},
  numbers=left,
  numberstyle=\tiny,
  numbersep=5pt,
  xleftmargin=10pt,
  framexleftmargin=10pt,
  escapechar=|!,
}

\definecolor{keywordcolor}{rgb}{0.7, 0.1, 0.1}   %
\definecolor{tacticcolor}{rgb}{0.0, 0.1, 0.6}    %
\definecolor{commentcolor}{rgb}{0.4, 0.4, 0.4}   %
\definecolor{symbolcolor}{rgb}{0.0, 0.1, 0.6}    %
\definecolor{sortcolor}{rgb}{0.1, 0.5, 0.1}      %
\definecolor{attributecolor}{rgb}{0.7, 0.1, 0.1} %


\lstdefinelanguage{lean}{
  morekeywords={def,theorem,example,axiom,constant,inductive,structure,namespace,variable,universe,forall,exists},
  sensitive=true,
  morecomment=[l]{--},
  morestring=[b]{"},
  escapeinside={<@}{@>},
}

\lstdefinestyle{leanstyle}{
  language=lean,
  % basicstyle=\ttfamily,
  mathescape=true,
  commentstyle=\color{green!50!black},
  keywordstyle=\color{blue},
  stringstyle=\color{orange},
  showstringspaces=false,
  breaklines=true,
  frame=single,
  numbers=left,
  numberstyle=\tiny,
  numbersep=5pt,
  xleftmargin=10pt,
  framexleftmargin=10pt,
}



% Commands for notation

%left-right bracket 
\newcommand{\lr}[1]{\left( #1 \right)}
%letf-right curly bracket 
\newcommand{\lrcb}[1]{\left\{ #1 \right\}}

\newcommand{\bigO}{\mathcal{O}}
\newcommand{\para}[1]{\smallskip\noindent\textbf{\mdseries\textsc{#1.}}}
\newcommand{\dom}{\textup{\textbf{dom}}}

% special letters for reals, rationals, integrals 
\newcommand{\RP}{\mathbb{R^+}}
\newcommand{\nn}{\mathbf{n}}
\newcommand{\mm}{\mathbf{m}}
\newcommand{\RR}{\mathbb{R}}
\newcommand{\QQ}{\mathbb{Q}}
\newcommand{\NN}{\mathbb{N}}
\newcommand{\EE}{\mathbb{E}}
\newcommand{\ttt}{\mathcal{T}}
\newcommand{\ifthen}{\textcolor{blue}{\texttt{if-then}}}
\newcommand{\rex}{\mathcal{R}\texttt{-expression}}
\newcommand{\mex}{\mathcal{M}\texttt{-monomial}}
\newcommand{\QPoly}{Q\text{-Polynomial}}
\newcommand{\mtrans}{M\text{-Transformation}}

% From: https://tex.stackexchange.com/questions/323297/typing-block-matrices-with-zero-blocks-and-seperators
% \newcommand{\bigmat}[1]{\mbox{\normalfont\bfseries #1}}
% \newcommand{\rvline}{\hspace*{-\arraycolsep}\vline\hspace*{-\arraycolsep}}


\newcommand{\SAT}[1]{\textit{SAT}(#1)}
\newcommand{\monoid}{\text{monoid}}
\newcommand{\degreemath}{\textit{degree}}
\renewcommand\thmcontinues[1]{Continued}

\newcommand{\ifthenblock}[2]{\textcolor{blue}{\texttt{if}} \; (\textcolor{red}{#1}) \; \textcolor{blue}{\texttt{then}} \; \textcolor{green!50!black}{#2}}


% \newcommand{\todo}[1]{\textcolor{red}{\textbf{TODO #1}}}

\newcommand{\Rstep}{\underset{R}{\rightarrow}}
\newcommand{\Rstepstar}{\underset{R}{\overset{*}{\rightarrow}}} 

% for cells in the experimental section 
% \newcommand{\success}{\textcolor{green!50!black}{\textbf{+}}}
\newcommand{\fail}{\textcolor{red}{\textbf{-}}}
\newcommand{\missed}{\textcolor{blue}{\textbf{missed}}}
\newcommand{\success}{\textcolor{gruen}{\textbf{\boldcheckmark}}}
\newcommand{\unsat}{\textcolor{rot}{\textbf{UNSAT}}}
\newcommand{\tl}{\textcolor{rot}{\textbf{TL}}}
\newcommand{\wa}{\textcolor{rot}{\textbf{WA}}}
\newcommand{\ns}{\textcolor{rot}{\textbf{NS}}}
\newcommand{\unknown}{\textcolor{rot}{\textbf{NS}}}
\newcommand{\outofmem}{\textcolor{rot}{\textbf{ML}}}
\newcommand{\sat}{\textsl{SAT}}

\newcommand{\valuation}{val}
\newcommand{\val}{\valuation}
\newcommand{\vars}{\mathbb{V}}
\newcommand{\tvars}{{\mathbb{T}}}
\newcommand{\tval}{\valuation_\tvars}
\newcommand{\locs}{\mathbb{L}}
\newcommand{\loc}{\ell}
\newcommand{\transitions}{\mathcal{T}}
\newcommand{\transition}{\tau}
\newcommand{\invariant}{\mathbb{I}}
\newcommand{\cutset}{\mathcal{C}}
\newcommand{\state}{\sigma}

\newcommand{\linearexample}{\textit{VotingContract }}
\newcommand{\nonlinearexample}{\textit{NestedLoop }}


\newcommand{\qflra}{\text{QF\_LRA}\xspace}
\newcommand{\qfuf}{\text{QF\_UF}\xspace}

\newcommand{\edrat}{\textit{e}\text{DRAT}\xspace}
\newcommand{\drat}{\text{DRAT}\xspace}
\newcommand{\dimacs}{\text{DIMACS}\xspace}
\newcommand{\prop}[0]{\mathcal{P}\xspace}
\newcommand{\cvcv}{\textsc{cvc5}\xspace}
\newcommand{\tick}{\textcolor{green!50!black}{\checkmark}\xspace}
\newcommand{\noresult}{\textcolor{red!80!blue}{NA}}

\newcommand{\red}[1]{\textcolor{red!80!blue}{#1}}
\newcommand{\green}[1]{\textcolor{Green}{#1}}
\newcommand{\Gray}[1]{\textcolor{Gray}{#1}}

\newcommand{\Sat}[0]{\text{SAT}\xspace}
\newcommand{\leanvalido}[0]{\textsc{LeanValido}\xspace}
\newcommand{\valido}[0]{\textsc{Valido}\xspace}

\newcommand{\drattrim}{\texttt{DRAT-trim}}



\newcommand{\algovar}[1]{\textit{#1}}
\newcommand{\funcname}[1]{\textcolor{Blue}{\text{#1}}}
\newcommand{\toolname}[1]{\textcolor{Blue}{\text{#1}}}
\newcommand{\funcnamelean}[1]{\textcolor{Blue}{\texttt{#1}}}
\newcommand{\intextfunc}[1]{\texttt{#1}}

\newcommand{\lstcomment}[1]{\hfill \textcolor{Gray}{\textit{#1}}}


% For introduction of thesis

\tcbuselibrary{skins, breakable}

\newtcolorbox{doublebox}{
  enhanced,
  % breakable,
  colback=lightgray!10,     % light gray background
  % colframe=black,           % black border
  % boxrule=0.5pt,            % border thickness
  % frame style={double},     % double outer lines
  % sharp corners,
  % left=6pt, right=6pt, top=6pt, bottom=6pt,
}


% For nicer font for \textsc and latin quote

% \usepackage{palatino} % or mathpazo for Palatino

% Define a macro for Latin quotes in all caps
\newcommand{\latinquote}[1]{\textsc{\MakeUppercase{#1}}}