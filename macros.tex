%%%%%%%%%%%%%%%%%%%%%%%%%%%%%%%%%%%%%%%%%%%%%%%%%%%%%%%%%%%%%%%%%%%%%%%%%
%                                                                       %
% macros.tex: Consolidated macros for PhD thesis                        %
%                                                                       %
%%%%%%%%%%%%%%%%%%%%%%%%%%%%%%%%%%%%%%%%%%%%%%%%%%%%%%%%%%%%%%%%%%%%%%%%%

%%%%%%%%%%%%%%%%%%%%%%%%%%%%%%%%%%%%%%%%%%%%%%%%%%%%%%%%%%%%%%%%%%%%%%%%%
% SHARED PACKAGES AND SETTINGS                                          %
%%%%%%%%%%%%%%%%%%%%%%%%%%%%%%%%%%%%%%%%%%%%%%%%%%%%%%%%%%%%%%%%%%%%%%%%%

% For better font rendering
% \usepackage{lmodern}
% For citations and references (adjust style as needed)
%HS: Add following for fullcite
%\usepackage[backend=biber, style=authoryear]{biblatex}
%\addbibresource{refs.bib} % Replace with your .bib file name
%\usepackage{natbib}
% make references clickable 
% \usepackage{unicode-math}
% \setmathfont{Latin Modern Math} % or another math font if you prefer

\usepackage[cal=cm,bb=ams,scr=boondoxo]{mathalfa} % One family for various math families
\usepackage{bibentry}
\usepackage[hidelinks, pdfpagelabels]{hyperref}
\usepackage{makecell}
\usepackage{rotating}
\usepackage{mathtools}
% \usepackage{amsfonts}
\usepackage{nicefrac}
\usepackage{float}
\usepackage{wrapfig}
\usepackage{amscd}
\usepackage{amsthm}
\usepackage{verbatim}
\usepackage{tikz}
\usepackage{pgfplots}
\pgfplotsset{compat=1.18}

\usetikzlibrary{calc, matrix, shapes, arrows, positioning, decorations.pathreplacing, backgrounds}
\usetikzlibrary{shapes.geometric, fit}
\usetikzlibrary{arrows.meta}
\usepackage[many]{tcolorbox}
\usepackage{color,colortbl}
\usepackage{adjustbox}
\usepackage{changepage}
\usepackage{framed}
\tikzstyle{box_nofill}=[rectangle,draw]
\tikzstyle{box_fill1}=[rectangle,draw,fill=asparagus!50,minimum size=1.4em]
\tikzstyle{box_fill2}=[rectangle,draw,fill=babyblueeyes!50,minimum size=1.4em]
% \usepackage{subcaption}
% \usepackage{pgfplots}
% \usepackage{thmtools}

% NOTE: This thesis uses the `algorithm2e` pseudocode style (\KwIn/\KwResult/\ForEach/...)
% which conflicts with the `algorithm` + `algpseudocode` combination.
% Keep only one family enabled.
%\usepackage{algorithm}
%\usepackage[noend]{algpseudocode}

\usepackage[utf8]{inputenc}
\usepackage[T1]{fontenc}
\usepackage{xspace}

\usepackage{amssymb}
\usepackage{stmaryrd}
% The stmaryrd font family doesn't provide a bold series; declare a
% substitution so LaTeX won't warn when a bold series is requested.
% If the stmaryrd FD file is available, load it first and then declare
% the bold->medium substitution for the stmry family. This avoids the
% "font family unknown" error when the FD file hasn't been read yet.
% Define the stmry math symbol font and map its bold version to the
% medium series so that bold math (e.g. via \bm) won't request a
% non-existent bold font.
\DeclareSymbolFont{stmry}{U}{stmry}{m}{n}
\SetSymbolFont{stmry}{bold}{U}{stmry}{m}{n}

% Suppress Overfull/Underfull \hbox and \vbox warnings (reduces noisy
% diagnostics in editors like VS Code). Adjust if you want looser
% control instead of fully silencing these messages.
% - \hfuzz: ignore overfull boxes smaller than this glue (in pt)
% - \hbadness: threshold for reporting underfull hbox (0..10000)
% - \vfuzz/\vbadness analogues for vertical boxes
% - \overfullrule: width of visible box in output; 0pt hides it
\hfuzz=100pt
\hbadness=10000
\vfuzz=100pt
\vbadness=10000
\overfullrule=0pt
\usepackage{etoolbox}
\usepackage{bm}
% \usepackage{cite}
%HS: Removing cite for hkust thesis template
\usepackage{textcomp}
\usepackage{tabto}
\usepackage{thm-restate}
\usepackage{blkarray}
\usepackage{refcount}

% \usepackage{lineno}
% \linenumbers

\usepackage{caption}

\usepackage{xcolor}
\def\BibTeX{{\rm B\kern-.05em{\sc i\kern-.025em b}\kern-.08em
    T\kern-.1667em\lower.7ex\hbox{E}\kern-.125emX}}


\usepackage{svg}

\usepackage[ruled,linesnumbered]{algorithm2e}

% Compatibility: some pseudocode uses \Comment{...} (algpseudocode-style) while
% the thesis otherwise uses algorithm2e keywords (\KwIn/\KwResult/\ForEach).
% Map \Comment to an algorithm2e end-of-line comment.
\providecommand{\Comment}[1]{\tcp*[f]{#1}}

\definecolor{gray}{rgb}{0.3, 0.3, 0.3}

% \SetCommentSty{\textcolor{gray}{\#}} % Gray color for comments
% \SetKwComment{tcp}{\color{gray}$\blacktriangleright$ }{}

% % Redefine comment style
% \SetCommentSty{textit} % Italicize comment
% \SetCommentSty{color{gray}} % Set color to gray
% \SetKwComment{Comment}{$\blacktriangleright$\ }{} % Use black triangle



% \usepackage{setspace}
\usepackage{listings}
\usepackage{pgf}
\usepackage{graphicx}
% \usepackage{subfig}
\usepackage{dashbox}
\usepackage{float}
\usepackage{syntax}
% \usepackage{hyperref}
\usepackage{hhline, booktabs}
\usepackage{array,multirow}	
\usepackage{tablefootnote}
% \usepackage{subcaption}
% \usepackage{longtable} % For long tables
\usepackage{cleveref}
% \usepackage{mathrsfs}
\usepackage{subcaption} 




\setlength{\fboxsep}{1pt}
\setlength{\dashlength}{4pt}
\newcommand{\codetemplate}[1]{\dbox{$#1$}}
\newcommand{\Hole}[1]{\dbox{$#1$}}
\newcommand{\errorRejected}{\textit{Failed}}
\newcommand{\errorRose}{\textit{Failed}}
\newcommand{\errorUnsat}{\textit{Unsat}}
\newcommand{\errorTimeout}{\textit{Timed out}}
\usepackage{paralist}
\usepackage{lscape}

\newcommand{\hit}[1]{\textcolor{red}{\textbf{Hitarth: }\textit{#1}}}

\usepackage{amsmath}
% \usepackage{thmtools}

\usepackage{thmtools}
\usepackage{thm-restate}


\theoremstyle{plain}
% \newtheorem{example}{Example}
% \newtheorem{definition}{Definition}
\newtheorem{namedtheorem}{Theorem}
\newtheorem{namedlemma}{Lemma}

\newtheoremstyle{noparens}% cf. p. 10 of user guide of 'amsthm' package
    {}{}{\itshape}{}%
    {\bfseries}{.}{ }%
    {\thmname{#1}\thmnumber{ #2}\thmnote{ {(\mdseries #3)}}}

\theoremstyle{noparens}
\newtheorem{theorem}[namedtheorem]{Theorem}
\newtheorem{block2024crypto}{Open problem}
\newtheorem{proposition}{Proposition}
\newtheorem{conjecture}{Conjecture}
\newtheorem*{fact*}{Fact}
\newtheorem{excont}{Example}
\newtheorem{corollary}{Corollary}
\newtheorem{lemma}{Lemma}
\newtheorem{claim}{Claim}
\newtheorem{definition}{Definition}
\newtheorem{example}{Example}
\theoremstyle{remark}
\newtheorem{remark}{Remark}
\newtheorem*{remark*}{Remark}


\renewcommand{\theexcont}{\theexample}
% \let\example\relax
% \declaretheorem[style=remark]{Example}

% CONFLICT: Multiple definitions of \paragraph exist across files
% macros-hkust.tex version (commented out):
% \renewcommand{\paragraph}[1]{\smallskip\noindent\textbf{\emph{#1}}} % Redefine paragraph to be bold and italicized
% options-macros.tex version:
% \renewcommand{\paragraph}[1]{\smallskip\noindent\textsc{{#1.}}}
% mining-ethereum-macros.tex version:
% \renewcommand{\paragraph}[1]{\smallskip\noindent\textbf{\textsc{#1.}}}
% Using mining version (bold+smallcaps) as default - CHECK THIS:
\renewcommand{\paragraph}[1]{\smallskip\noindent\textbf{\textsc{#1.}}}

\lstset
{ %Formatting for code in appendix
    numbers=left,
    stepnumber=1,
    showstringspaces=false,
    tabsize=1,
    breaklines=true,
    breakatwhitespace=false,
}


\usepackage{listings}
\definecolor{codegreen}{rgb}{0,0.6,0}
\definecolor{codegray}{rgb}{0.5,0.5,0.5}
\definecolor{codepurple}{rgb}{0.58,0,0.82}
\definecolor{backcolour}{rgb}{0.95,0.95,0.92}
\definecolor{gruen}{rgb}{0.0, 0.5, 0.0}
\definecolor{rot}{rgb}{1.0, 0.13, 0.32}

% \lstdefinestyle{C}{
% morekeywords={then}
% }

% \lstset{
%     keywords={then},
%     keywordstyle=\color{red}
% }    

\lstset{
basicstyle=\ttfamily,columns=flexible,frame=single,framerule=0pt,%
	%backgroundcolor=\color{gray!20},%
	xleftmargin=\fboxsep,%
	xrightmargin=\fboxsep,
	language=[LaTeX]TeX,%
	numbers=left,
	keywordstyle=\color{blue},%
	texcsstyle=*\color{red}\bfseries,%
	texcs={end,begin,documentclass,graphicspath},%
	mathescape=false,escapechar=|,%
	literate={<B>}{\textcolor{blue}{\string\usepackage}}1
	{\{ }{\textcolor{blue}{\{}}1
	{\}}{\textcolor{blue}{\}}}1
	{[}{\textcolor{blue}{[}}1     
	{]}{\textcolor{blue}{]}}1
        {then}{\textcolor{blue}{then }}1
}
\lstset{emph={%  
		@prog, @real, @pre, @post, @invariant, @macro, @var, @recurrence, %
	},emphstyle={\color{red}\bfseries}%
}%

% \usepackage{xcolor}


\usepackage{enumitem}
\newlist{mycompactitem}{itemize}{3} % 3 is max-depth
\setlist[mycompactitem]{leftmargin=0em,label=\textbullet, nosep}
\newlist{mycompactenum}{enumerate}{3} % 3 is max-depth
\setlist[mycompactenum]{leftmargin=.6em,label=\textnormal{(\arabic*)},nosep}

\newcommand{\aoverb}[2]{$\substack{\text{#1} \\ \text{#2}}$}


% \lstset{basicstyle=\footnotesize\ttfamily,columns=flexible,frame=single,framerule=0pt,%
% 	%backgroundcolor=\color{gray!20},%
% 	xleftmargin=\fboxsep,%
% 	xrightmargin=\fboxsep,
% 	language=[LaTeX]TeX,%
% 	numbers=left,
% 	keywordstyle=\color{blue},%
% 	texcsstyle=*\color{red}\bfseries,%
% 	texcs={end,begin,documentclass,graphicspath},%
% 	mathescape=false,escapechar=|,%
% 	literate={<B>}{\textcolor{blue}{\string\usepackage}}1
% 	{\{ }{\textcolor{blue}{\{}}1
% 	{\}}{\textcolor{blue}{\}}}1
% 	{[}{\textcolor{blue}{[}}1     
% 	{]}{\textcolor{blue}{]}}1
% }
% \lstset{emph={%  
% 		@prog, @real, @pre, @post, @invariant, @macro%
% 	},emphstyle={\color{red}\bfseries}%
% }%


\usepackage{pdfrender}
\newcommand*{\boldcheckmark}{%
	\textpdfrender{
		TextRenderingMode=FillStroke,
		LineWidth=1pt, % half of the line width is outside the normal glyph
	}{\checkmark}%
}




\lstdefinestyle{smtlib-style}
{
  language=Lisp,
  frame=single,
  % basicstyle=\ttfamily,
  keywordstyle = [1]{\color{Gray}},
  keywordstyle = [2]{\color{Blue}},
  otherkeywords = {;,<<,>>,++},
  morekeywords = [1]{;},
  keywords = [2]{define-fun, define-literal, define-let, define, declare-sort, declare-fun, LINE, (, )},
  numbers=left,
  numberstyle=\tiny,
  numbersep=5pt,
  xleftmargin=10pt,
  framexleftmargin=10pt,
  escapechar=|!,
}

\definecolor{keywordcolor}{rgb}{0.7, 0.1, 0.1}   %
\definecolor{tacticcolor}{rgb}{0.0, 0.1, 0.6}    %
\definecolor{commentcolor}{rgb}{0.4, 0.4, 0.4}   %
\definecolor{symbolcolor}{rgb}{0.0, 0.1, 0.6}    %
\definecolor{sortcolor}{rgb}{0.1, 0.5, 0.1}      %
\definecolor{attributecolor}{rgb}{0.7, 0.1, 0.1} %


\lstdefinelanguage{lean}{
  morekeywords={def,theorem,example,axiom,constant,inductive,structure,namespace,variable,universe,forall,exists},
  sensitive=true,
  morecomment=[l]{--},
  morestring=[b]{"},
  escapeinside={<@}{@>},
}

\lstdefinestyle{leanstyle}{
  language=lean,
  % basicstyle=\ttfamily,
  mathescape=true,
  commentstyle=\color{green!50!black},
  keywordstyle=\color{blue},
  stringstyle=\color{orange},
  showstringspaces=false,
  breaklines=true,
  frame=single,
  numbers=left,
  numberstyle=\tiny,
  numbersep=5pt,
  xleftmargin=10pt,
  framexleftmargin=10pt,
}



% Commands for notation

%left-right bracket 
\newcommand{\lr}[1]{\left( #1 \right)}
%letf-right curly bracket 
\newcommand{\lrcb}[1]{\left\{ #1 \right\}}

% \newcommand{\bigO}{\mathcal{O}}
\newcommand{\para}[1]{\smallskip\noindent\textbf{\mdseries\textsc{#1.}}}
\newcommand{\dom}{\textup{\textbf{dom}}}

% Additional shared macros (kept if already defined elsewhere)
\providecommand*{\serg}[1]{\textcolor{blue}{#1}}

\providecommand{\mask}{\text{m}}
\providecommand{\maxcap}{C_{\text{max}}}
\providecommand{\dptargmax}{\text{dpargmax}}

\providecommand{\econf}{E_c}
\providecommand{\edep}{E_d}
\providecommand{\valid}{\text{valid}}
\providecommand{\concat}{\circ}
\providecommand{\suff}{\text{suff}}
\providecommand{\feas}{\mathcal{F}}

% make first parameter optional:
\providecommand{\clneig}[2][]{\mathcal{N}_{#1}[#2]}

\providecommand{\tdp}{\text{tdp}}
\providecommand{\txs}{\textsf{TX}}
\providecommand{\sz}{\sigma}
\newcommand{\cfee}{\phi}


% special letters for reals, rationals, integrals 
% \newcommand{\RP}{\mathbb{R^+}}
\newcommand{\nn}{\mathbf{n}}
\newcommand{\mm}{\mathbf{m}}
\newcommand{\RR}{\mathbb{R}}
\newcommand{\QQ}{\mathbb{Q}}
\newcommand{\NN}{\mathbb{N}}
% \newcommand{\EE}{\mathbb{E}}
\newcommand{\ttt}{\mathcal{T}}
\newcommand{\ifthen}{\textcolor{blue}{\texttt{if-then}}}
\newcommand{\rex}{\mathcal{R}\texttt{-expression}}
\newcommand{\mex}{\mathcal{M}\texttt{-monomial}}
\newcommand{\QPoly}{Q\text{-Polynomial}}
\newcommand{\mtrans}{M\text{-Transformation}}

% From: https://tex.stackexchange.com/questions/323297/typing-block-matrices-with-zero-blocks-and-seperators
% \newcommand{\bigmat}[1]{\mbox{\normalfont\bfseries #1}}
% \newcommand{\rvline}{\hspace*{-\arraycolsep}\vline\hspace*{-\arraycolsep}}


\newcommand{\SAT}[1]{\textit{SAT}(#1)}
\newcommand{\monoid}{\text{monoid}}
\newcommand{\degreemath}{\textit{degree}}
\renewcommand\thmcontinues[1]{Continued}

\newcommand{\ifthenblock}[2]{\textcolor{blue}{\texttt{if}} \; (\textcolor{red}{#1}) \; \textcolor{blue}{\texttt{then}} \; \textcolor{green!50!black}{#2}}


% CONFLICT: \todo has different definitions in options-macros.tex and mining-ethereum-macros.tex
% options-macros.tex version:
% \newcommand{\todo}{\textcolor{red}{\textbf{TODO }}}
% \renewcommand{\todo}[1]{\textcolor{red}{TODO: #1}}
% mining-ethereum-macros.tex has it commented out
% Using options-macros.tex version - CHECK THIS:
\newcommand{\todo}{\textcolor{red}{\textbf{TODO }}}
% \renewcommand{\todo}[1]{\textcolor{red}{TODO: #1}}

\newcommand{\Rstep}{\underset{R}{\rightarrow}}
\newcommand{\Rstepstar}{\underset{R}{\overset{*}{\rightarrow}}} 

% for cells in the experimental section 
% \newcommand{\success}{\textcolor{green!50!black}{\textbf{+}}}
\newcommand{\fail}{\textcolor{red}{\textbf{-}}}
\newcommand{\missed}{\textcolor{blue}{\textbf{missed}}}
\newcommand{\success}{\textcolor{gruen}{\textbf{\boldcheckmark}}}
\newcommand{\unsat}{\textcolor{rot}{\textbf{UNSAT}}}
\newcommand{\tl}{\textcolor{rot}{\textbf{TL}}}
\newcommand{\wa}{\textcolor{rot}{\textbf{WA}}}
\newcommand{\ns}{\textcolor{rot}{\textbf{NS}}}
\newcommand{\unknown}{\textcolor{rot}{\textbf{NS}}}
\newcommand{\outofmem}{\textcolor{rot}{\textbf{ML}}}
\newcommand{\sat}{\textsl{SAT}}

\newcommand{\valuation}{val}
\newcommand{\val}{\valuation}
\newcommand{\vars}{\mathbb{V}}
\newcommand{\tvars}{{\mathbb{T}}}
\newcommand{\tval}{\valuation_\tvars}
\newcommand{\locs}{\mathbb{L}}
\newcommand{\loc}{\ell}
\newcommand{\transitions}{\mathcal{T}}
\newcommand{\transition}{\tau}
\newcommand{\invariant}{\mathbb{I}}
\newcommand{\cutset}{\mathcal{C}}
\newcommand{\state}{\sigma}

\newcommand{\linearexample}{\textit{VotingContract }}
\newcommand{\nonlinearexample}{\textit{NestedLoop }}


\newcommand{\qflra}{\text{QF\_LRA}\xspace}
\newcommand{\qfuf}{\text{QF\_UF}\xspace}

\newcommand{\edrat}{\textit{e}\text{DRAT}\xspace}
\newcommand{\drat}{\text{DRAT}\xspace}
\newcommand{\dimacs}{\text{DIMACS}\xspace}
\newcommand{\prop}[0]{\mathcal{P}\xspace}
\newcommand{\cvcv}{\textsc{cvc5}\xspace}
\newcommand{\tick}{\textcolor{green!50!black}{\checkmark}\xspace}
\newcommand{\noresult}{\textcolor{red!80!blue}{NA}}

\newcommand{\red}[1]{\textcolor{red!80!blue}{#1}}
\newcommand{\green}[1]{\textcolor{Green}{#1}}
\newcommand{\Gray}[1]{\textcolor{Gray}{#1}}

\newcommand{\Sat}[0]{\text{SAT}\xspace}
\newcommand{\leanvalido}[0]{\textsc{LeanValido}\xspace}
\newcommand{\valido}[0]{\textsc{Valido}\xspace}

\newcommand{\drattrim}{\texttt{DRAT-trim}}



\newcommand{\algovar}[1]{\textit{#1}}
\newcommand{\funcname}[1]{\textcolor{Blue}{\text{#1}}}
\newcommand{\toolname}[1]{\textcolor{Blue}{\text{#1}}}
\newcommand{\funcnamelean}[1]{\textcolor{Blue}{\texttt{#1}}}
\newcommand{\intextfunc}[1]{\texttt{#1}}

\newcommand{\lstcomment}[1]{\hfill \textcolor{Gray}{\textit{#1}}}


% For introduction of thesis

\tcbuselibrary{skins, breakable}

\newtcolorbox{doublebox}{
  enhanced,
  % breakable,
  colback=lightgray!10,     % light gray background
  % colframe=black,           % black border
  % boxrule=0.5pt,            % border thickness
  % frame style={double},     % double outer lines
  % sharp corners,
  % left=6pt, right=6pt, top=6pt, bottom=6pt,
}


% For nicer font for \textsc and latin quote

% \usepackage{palatino} % or mathpazo for Palatino

% Define a macro for Latin quotes in all caps
\newcommand{\latinquote}[1]{\textsc{\MakeUppercase{#1}}}

%%%%%%%%%%%%%%%%%%%%%%%%%%%%%%%%%%%%%%%%%%%%%%%%%%%%%%%%%%%%%%%%%%%%%%%%%
%                                                                       %
% OPTIONS PAPER SPECIFIC MACROS                                         %
%                                                                       %
%%%%%%%%%%%%%%%%%%%%%%%%%%%%%%%%%%%%%%%%%%%%%%%%%%%%%%%%%%%%%%%%%%%%%%%%%

\newcommand{\contract}{\mathcal{C}}
\newcommand{\wrapped}{\mathcal{W}}
\newcommand{\parties}{\mathcal{P}}
\newcommand{\party}{p}
\newcommand{\alice}{\textsc{Alice}}
\newcommand{\bob}{\textsc{Bob}}
\newcommand{\ingrid}{\textsc{Ingrid}}
\newcommand{\paul}{\textsc{Paul}}

% CONFLICT: \gamma redefined in options-macros.tex as 'gb'
% Original \gamma is a Greek letter, this redefines it - CHECK THIS:
% \renewcommand{\gamma}{gb}


%%%%%%%%%%%%%%%%%%%%%%%%%%%%%%%%%%%%%%%%%%%%%%%%%%%%%%%%%%%%%%%%%%%%%%%%%
%                                                                       %
% MINING (ETHEREUM) PAPER SPECIFIC MACROS                               %
%                                                                       %
%%%%%%%%%%%%%%%%%%%%%%%%%%%%%%%%%%%%%%%%%%%%%%%%%%%%%%%%%%%%%%%%%%%%%%%%%

% use tikz 
\usepackage{tikz,forest}

\usepackage{xparse}
\usepackage{multirow}
\usepackage{bbding} 
% \usepackage{todonotes}  % COMMENTED: conflicts with \todo command defined elsewhere

\newcommand{\ilp}[1]{\llbracket #1 \rrbracket}

% \forestset{
% default preamble={
% before typesetting nodes={
%   !r.replace by={[, coordinate, append]}
% },  
% where n children=0{
%   tier=word,
% }{  
%   %diamond, aspect=2,
% },  
% where level=0{
%   s sep'-=0.3cm,
% }{
%   if n=1{
%     edge label={node[pos=.3, above] {Y}},
%   }{  
%     edge label={node[pos=.3, above] {N}},
%   }   
% },  
% for tree={
%   edge+={thick, -Latex},
%   s sep'+=0.6cm,
%   draw,
%   thick,
%   edge path'={ (!u) -| (.parent)},
%   align=center,
% }   
% }
% }


\forestset{
default preamble={
where n children=0{
  tier=word,
}{  
  diamond, aspect=3.5,
},
where level=0{}{if n=1{
edge label={node[pos=.5, above=0.1, left] {y}},
}{  
edge label={node[pos=.5, above=0.1, right] {n}},
}
},
% for tree={
%   % edge+={thick, -Latex},
%   s sep'+=0.3cm,
%   draw,
%   thin,
%   edge path'={ (!u) - (.parent)},
%   % align=center,  
% }
}
}

% \forestset{
% default preamble={
% before typesetting nodes={
%   !r.replace by={[, coordinate, append]}
% },  
% where n children=0{
%   tier=word,
% }{  
%   %diamond, aspect=2,
% },  
% where level=0{
%   s sep'-=0.3cm,
% }{
%   if n=1{
%     edge label={node[pos=.3, above] {Y}},
%   }{  
%     edge label={node[pos=.3, above] {N}},
%   }   
% },  
% for tree={
%   edge+={thick, -Latex},
%   s sep'+=0.6cm,
%   draw,
%   thick,
%   edge path'={ (!u) -| (.parent)},
%   align=center,
% }   
% }
% }


\lstdefinelanguage{Solidity}{
	keywords=[1]{anonymous, assembly, assert, balance, break, call, callcode, case, catch, class, constant, continue, constructor, contract, debugger, default, delegatecall, delete, do, else, emit, event, experimental, export, external, false, finally, for, function, gas, if, implements, import, in, indexed, instanceof, interface, internal, is, length, library, log0, log1, log2, log3, log4, memory, modifier, new, payable, pragma, private, protected, public, pure, push, require, return, returns, revert, selfdestruct, send, solidity, storage, struct, suicide, super, switch, then, this, throw, transfer, true, try, typeof, using, value, view, while, with, addmod, ecrecover, keccak256, mulmod, ripemd160, sha256, sha3}, % generic keywords including crypto operations
	keywordstyle=[1]\color{blue}\bfseries,
	keywords=[2]{address, bool, byte, bytes, bytes1, bytes2, bytes3, bytes4, bytes5, bytes6, bytes7, bytes8, bytes9, bytes10, bytes11, bytes12, bytes13, bytes14, bytes15, bytes16, bytes17, bytes18, bytes19, bytes20, bytes21, bytes22, bytes23, bytes24, bytes25, bytes26, bytes27, bytes28, bytes29, bytes30, bytes31, bytes32, enum, int, int8, int16, int24, int32, int40, int48, int56, int64, int72, int80, int88, int96, int104, int112, int120, int128, int136, int144, int152, int160, int168, int176, int184, int192, int200, int208, int216, int224, int232, int240, int248, int256, mapping, string, uint, uint8, uint16, uint24, uint32, uint40, uint48, uint56, uint64, uint72, uint80, uint88, uint96, uint104, uint112, uint120, uint128, uint136, uint144, uint152, uint160, uint168, uint176, uint184, uint192, uint200, uint208, uint216, uint224, uint232, uint240, uint248, uint256, var, void, ether, finney, szabo, wei, days, hours, minutes, seconds, weeks, years},	% types; money and time units
	keywordstyle=[2]\color{teal}\bfseries,
	keywords=[3]{block, blockhash, coinbase, difficulty, gaslimit, number, timestamp, msg, data, gas, sender, sig, value, now, tx, gasprice, origin},	% environment variables
	keywordstyle=[3]\color{violet}\bfseries,
	identifierstyle=\color{black},
	sensitive=true,
	comment=[l]{//},
	morecomment=[s]{/*}{*/},
	commentstyle=\color{gray}\ttfamily,
	stringstyle=\color{red}\ttfamily,
	morestring=[b]',
	morestring=[b]"
}

% algo2e 
% \usepackage[ruled,vlined]{algorithm2e}  % COMMENTED: conflicts with algorithm package in shared section
% \usepackage{algorithm}
% \usepackage{algorithmic}

\newcommand{\before}{\prec}
\newcommand{\has}{\diamondsuit}
\newcommand{\atleast}{\texttt{atleast}}

\theoremstyle{plain}%
% \newtheorem{theorem}{Theorem}%  meant for continuous numbers
% \newtheorem{theorem}{Theorem}[section]% meant for sectionwise numbers
% optional argument [theorem] produces theorem numbering sequence instead of independent numbers for Proposition
% \newtheorem{proposition}[theorem]{Proposition}

% \newtheorem{proposition}{Proposition}% to get separate numbers for theorem and proposition etc.
% \newtheorem{lemma}[theorem]{Lemma}

\theoremstyle{thmstyletwo}%
% \newtheorem{example}{Example}%
% \newtheorem{remark}{Remark}%

\newtheorem{notation}[theorem]{Notation}%

% \theoremstyle{thmstylethree}%
% \newtheorem{definition}{Definition}%


% \newcommand{\fixme}[1]{{\textbf{\color{red}(fixme: #1)}}}

\newcommand{\tx}{\ensuremath{\textup{tx}}\xspace}
\newcommand{\nonce}{\ensuremath{\textup{nonce}}\xspace}
\newcommand{\addr}{\ensuremath{\textup{addr}}\xspace}
\newcommand{\hash}{\ensuremath{\textup{hash}}\xspace}
\newcommand{\txaddr}[1]{\ensuremath\tx(\text{#1})\xspace}
\newcommand{\nbhd}{\ensuremath{\textup{nbhd}}\xspace}
\newcommand{\compat}{\leftrightarrows}
\newcommand{\incompat}{\not\leftrightarrows}
\newcommand{\incompatSet}{\ensuremath{\textup{Incompat}}\xspace}
\newcommand{\feat}{\ensuremath{\textup{feat}}\xspace}
\newcommand{\resp}{\ensuremath{\textup{resp}}\xspace}
\newcommand{\subseq}{\preceq}
\newcommand{\Var}{\ensuremath{\textup{Var}}\xspace}


% % this allows to overload the \patch command to use \patch, \patch[\tx] and \patch[\tx][1]


% Command for half circle
\newcommand*\halfcirc[1][1ex]{%
  \begin{tikzpicture}
  \draw[fill] (0,0)-- (90:#1) arc (90:270:#1) -- cycle ;
  \draw (0,0) circle (#1);
  \end{tikzpicture}}

% Command for full circle
\newcommand*\fullcirc[1][1ex]{%
  \begin{tikzpicture}
  \draw[fill] (0,0) circle (#1);
  \end{tikzpicture}}

% Command for empty circle
\newcommand*\emptycirc[1][1ex]{%
  \begin{tikzpicture}
  \draw (0,0) circle (#1);
  \end{tikzpicture}}

% Just patch 
\newcommand{\mixedType}{\halfcirc[0.06]}
\NewDocumentCommand{\patch}{o o}{%
  \ensuremath{p^{\mixedType}%
  \IfValueT{#1}{_{#1\IfValueT{#2}{, #2}}}\xspace}}

% Strict patch with full circle
\newcommand{\strictType}{\fullcirc[0.06]}
\NewDocumentCommand{\patchS}{o o}{%
  \ensuremath{p^{\strictType}%
  \IfValueT{#1}{_{#1\IfValueT{#2}{, #2}}}\xspace}}

% Relaxed patch with empty circle
\newcommand{\relaxedType}{\emptycirc[0.06]}
\NewDocumentCommand{\patchR}{o o}{%
  \ensuremath{p^{\relaxedType}%
  \IfValueT{#1}{_{#1\IfValueT{#2}{, #2}}}\xspace}}

% Full set of just patches 
\NewDocumentCommand{\patchFull}{o o}{%
  \ensuremath{\textbf{P}^{\,\mixedType}%
  \IfValueT{#1}{_{#1\IfValueT{#2}{, #2}}}\xspace}}

% Strict patch with full circle
\NewDocumentCommand{\patchSFull}{o o}{%
  \ensuremath{\textbf{P}^{\,\strictType}%
  \IfValueT{#1}{_{#1\IfValueT{#2}{, #2}}}\xspace}}

% Relaxed patch with empty circle
\NewDocumentCommand{\patchRFull}{o o}{%
  \ensuremath{\textbf{P}^{\,\relaxedType}%
  \IfValueT{#1}{_{#1\IfValueT{#2}{, #2}}}\xspace}}



\newcommand\restr[2]{{% we make the whole thing an ordinary symbol
\left.\kern-\nulldelimiterspace % automatically resize the bar with \right
#1 % the function
\vphantom{\big|} % pretend it's a little taller at normal size
\right|_{#2} % this is the delimiter
}}

\newcommand{\gr}{\ensuremath{G}\xspace}

\newcommand{\patchnbhd}{\ensuremath{p}\xspace}
\newcommand{\tail}{\ensuremath{\textup{tail}}\xspace}
\newcommand{\head}{\ensuremath{\textup{head}}\xspace}
\newcommand{\core}{\ensuremath{\textup{core}}\xspace}

% for application of \forall and \exists
\newcommand{\dotapp}{\mathbf{\,.\,}}

% CONFLICT: \EE has different definitions
% macros-hkust.tex has: \newcommand{\EE}{\mathbb{E}} (commented out)
% mining-ethereum-macros.tex has: \newcommand{\EE}[1]{\mathbb{E}\left[#1\right]}
% Using mining version - CHECK THIS:
\newcommand{\EE}[1]{\mathbb{E}\left[#1\right]}
\newcommand{\EEs}[2]{\mathbb{E}_{#1}\left[#2\right]}
\newcommand{\PP}[1]{\mathbb{P}\left[#1\right]}
\newcommand{\PPs}[2]{\mathbb{P}_{#1}\left[#2\right]}

% CONFLICT: \bigO redefined
% macros-hkust.tex has: \newcommand{\bigO}{\mathcal{O}} (commented out)
% mining-ethereum-macros.tex has: \newcommand{\bigO}{\mathcal{O}}
% These are the same, using mining version - CHECK THIS:
\newcommand{\bigO}{\mathcal{O}}
\newcommand{\uniform}{\mathcal{U}}


\newcommand{\pool}{\ensuremath{{\textsf{Pool}}}\xspace}
\newcommand{\block}{\ensuremath{\textbf{\textup{block}}}\xspace}
\newcommand{\blocks}{\ensuremath{\textbf{\textup{Blocks}}}\xspace}
\newcommand{\blockopt}{\ensuremath{\textbf{\textup{block}}_{\textup{opt}}}\xspace}
\newcommand{\fee}{\ensuremath{\textup{fee}}\xspace}
\newcommand{\tip}{\ensuremath{\textup{tip}}\xspace}
\newcommand{\gas}{\ensuremath{\textsf{gas}}\xspace}
\newcommand{\feePerGas}{\ensuremath{\textup{fee/gas}}\xspace}
\newcommand{\tipPerGas}{\ensuremath{\tip/\gas}\xspace}


\newcommand{\dataset}{\ensuremath{\textup{Dataset}}\xspace}

\newcommand{\dpt}{\textup{dp}}
\newcommand{\pma}{\textup{PerfMatch}}

% CONFLICT: \RP redefined
% macros-hkust.tex has: \newcommand{\RP}{\mathbb{R^+}} (commented out)
% mining-ethereum-macros.tex has: \newcommand{\RP}{\mathcal{RP}}
% Using mining version - CHECK THIS:
\newcommand{\RP}{\mathcal{RP}}

\newcommand{\pw}{\textup{pw}}
\newcommand{\twi}{\textup{tw}}
\newcommand{\poly}{\textup{poly}}
\newcommand{\ma}{\textup{Match}}
\newcommand{\RM}{\mathcal{RM}}
\newcommand{\RI}{\mathcal{RI}}
\newcommand{\ind}{\textup{Ind}}
\newcommand{\SSt}{\textup{SStrees}}
\newcommand{\JoinN}{\textup{JoinNodes}}
\newcommand{\Schild}{\textup{SmallestChild}}

% \renewcommand{\printatom}[1]{%
% \fontsize{8pt}{10pt}\selectfont{\ensuremath{\mathsf{#1}}}}
% % reduce bond dimensions to match smaller fonts
% \setchemfig{
%   cram rectangle=false,
%   cram width=2.5pt,
%   cram dash width=0.4pt,
%   cram dash sep=1pt,
% atom sep=16pt,
%   bond offset=1pt,
%   double bond sep=2pt
% }



\newcommand{\ethsym}{%
  \!\raisebox{-0.7ex}{%
    \includegraphics[height=1.1em, trim={97 0 97 0}, clip]{figures/ethereum.pdf}%
  }%
}



% tikz block things 
% Define box dimensions as lengths
\newlength{\txBoxWidth}
\newlength{\txRowHeight}
\newlength{\txTotalHeight}
\setlength{\txBoxWidth}{0.55cm}
\setlength{\txRowHeight}{0.4cm}
\setlength{\txTotalHeight}{1.2cm}

\newcommand{\txBox}[7][.]{%
    % Parameters:
    % [#1]: border color (optional, defaults to black with '.')
    % #2: gas
    % #3: fee
    % #4: hash
    % #5: color
    % #6: column number (0-based)
    % #7: row number (0-based)
    \begin{scope}[shift={({\txBoxWidth*1.1*#6},{-\txTotalHeight*1.2*#7})}]
        % Main box with background color
        \fill[#5!15] (0,0) rectangle (\txBoxWidth,-\txTotalHeight);
        \draw[color=\ifx.#1black\else#1\fi] (0,0) rectangle (\txBoxWidth,-\txTotalHeight);
        
        % Horizontal lines for rows
        \draw[color=\ifx.#1black\else#1\fi] (0,-\txRowHeight) -- (\txBoxWidth,-\txRowHeight);
        \draw[color=\ifx.#1black\else#1\fi] (0,-2*\txRowHeight) -- (\txBoxWidth,-2*\txRowHeight);
        
        % Text content - left aligned values
        \node[anchor=west] at (-0.1,-0.5*\txRowHeight) {\scriptsize #2};
        \node[anchor=west] at (-0.1,-1.5*\txRowHeight) {\scriptsize #3};
        \node[anchor=west] at (-0.1,-2.5*\txRowHeight) {\scriptsize #4};
    \end{scope}
}
\newcommand{\txOutBoxOverlay}[4]{%
    % Parameters:
    % #1: color for the outer box
    % #2: dilation distance
    % #3: column number (0-based)
    % #4: row number (0-based)
    \begin{scope}[shift={({\txBoxWidth*1.1*#3-#2},{-\txTotalHeight*1.2*#4+#2})}]
        % Draw the overlay box with explicit width and height calculations
        \draw[color=#1, line width=1.5pt] 
            (0,0) rectangle (\txBoxWidth+2*#2,-\txTotalHeight-2*#2);
    \end{scope}
}


% Helper functions to define named coordinates for each box
\newcommand{\txDefineCorners}[2]{
    % #1: dx (column number), #2: dy (row number)
    % Upper row points (u)
    \coordinate (tx-#1-#2-ul) at (\txBoxWidth*1.1*#1, -\txTotalHeight*1.2*#2);
    \coordinate (tx-#1-#2-um) at (\txBoxWidth*1.1*#1 + 0.5*\txBoxWidth, -\txTotalHeight*1.2*#2);
    \coordinate (tx-#1-#2-ur) at (\txBoxWidth*1.1*#1 + \txBoxWidth, -\txTotalHeight*1.2*#2);
    
    % Middle row points (m)
    \coordinate (tx-#1-#2-ml) at (\txBoxWidth*1.1*#1, -\txTotalHeight*1.2*#2 - 0.5*\txTotalHeight);
    \coordinate (tx-#1-#2-mm) at (\txBoxWidth*1.1*#1 + 0.5*\txBoxWidth, -\txTotalHeight*1.2*#2 - 0.5*\txTotalHeight);
    \coordinate (tx-#1-#2-mr) at (\txBoxWidth*1.1*#1 + \txBoxWidth, -\txTotalHeight*1.2*#2 - 0.5*\txTotalHeight);
    
    % Lower row points (l)
    \coordinate (tx-#1-#2-ll) at (\txBoxWidth*1.1*#1, -\txTotalHeight*1.2*#2 - \txTotalHeight);
    \coordinate (tx-#1-#2-lm) at (\txBoxWidth*1.1*#1 + 0.5*\txBoxWidth, -\txTotalHeight*1.2*#2 - \txTotalHeight);
    \coordinate (tx-#1-#2-lr) at (\txBoxWidth*1.1*#1 + \txBoxWidth, -\txTotalHeight*1.2*#2 - \txTotalHeight);
}



% running example variavles 

\definecolor{cbUltra}{RGB}{81,81,255}      % Ultramarine Blue
\definecolor{cbMagenta}{RGB}{255,81,255}   % Magenta
\definecolor{cbCyan}{RGB}{81,255,255}      % Cyan
\definecolor{cbGreen}{RGB}{81,255,81}      % Green
\definecolor{cbYellow}{RGB}{255,255,81}    % Yellow
\definecolor{cbRed}{RGB}{255,81,81}        % Red
% disabled gray color
\definecolor{cbGrayFill}{RGB}{200,200,200}     % Gray
\definecolor{cbGrayOutline}{RGB}{230,230,230}     % Gray
% for disabled text color
\definecolor{cbGrayText}{RGB}{100,100,100}     % Gray


% Define transaction variables
\newcommand{\txheader}[2]{\txBox[white]{}{Gas $\gamma_i$}{Tip $\tau_i$}{white}{#1}{#2}}
\newcommand{\txa}[4]{\txBox{$\tx_1$}{#1}{#2}{cbUltra}{#3}{#4}}
\newcommand{\txb}[4]{\txBox{$\tx_2$}{#1}{#2}{cbMagenta}{#3}{#4}}
\newcommand{\txc}[4]{\txBox{$\tx_3$}{#1}{#2}{cbCyan}{#3}{#4}}
\newcommand{\txd}[4]{\txBox{$\tx_4$}{#1}{#2}{cbGreen}{#3}{#4}}
\newcommand{\txe}[4]{\txBox{$\tx_5$}{#1}{#2}{cbYellow}{#3}{#4}}
\newcommand{\txf}[4]{\txBox{$\tx_6$}{#1}{#2}{cbRed}{#3}{#4}}
\newcommand{\txres}[5]{\txBox[white]{$\pi_{#1}$}{$\Sigma\!\!:\,$#2}{$\Sigma\!\!:\,$#3}{white}{#4}{#5}}

% define important neighbors
\newcommand{\txaN}[4]{\txBox{$\tx_1$}{}{}{cbUltra}{#3}{#4}}
\newcommand{\txbN}[4]{\txBox{$\tx_2$}{}{}{cbMagenta}{#3}{#4}}
\newcommand{\txcN}[4]{\txBox{$\tx_3$}{}{}{cbCyan}{#3}{#4}}
\newcommand{\txdN}[4]{\txBox{$\tx_4$}{}{}{cbGreen}{#3}{#4}}
\newcommand{\txeN}[4]{\txBox{$\tx_5$}{}{}{cbYellow}{#3}{#4}}
\newcommand{\txfN}[4]{\txBox{$\tx_6$}{}{}{cbRed}{#3}{#4}}

% define off 
\newcommand{\txaOff}[4]{\txBox[cbGrayOutline]{\textcolor{gray}{$\tx_1$}}{}{}{cbGrayFill}{#3}{#4}}
\newcommand{\txbOff}[4]{\txBox[cbGrayOutline]{\textcolor{gray}{$\tx_2$}}{}{}{cbGrayFill}{#3}{#4}}
\newcommand{\txcOff}[4]{\txBox[cbGrayOutline]{\textcolor{gray}{$\tx_3$}}{}{}{cbGrayFill}{#3}{#4}}
\newcommand{\txdOff}[4]{\txBox[cbGrayOutline]{\textcolor{gray}{$\tx_4$}}{}{}{cbGrayFill}{#3}{#4}}
\newcommand{\txeOff}[4]{\txBox[cbGrayOutline]{\textcolor{gray}{$\tx_5$}}{}{}{cbGrayFill}{#3}{#4}}
\newcommand{\txfOff}[4]{\txBox[cbGrayOutline]{\textcolor{gray}{$\tx_6$}}{}{}{cbGrayFill}{#3}{#4}}



% define main transactions
\newcommand{\onDilatation}{0.4}
\newcommand{\txaOn}[4]{\txOutBoxOverlay{cbUltra}{\onDilatation}{#3}{#4}\txa{#1}{#2}{#3}{#4}}
\newcommand{\txbOn}[4]{\txOutBoxOverlay{cbMagenta}{\onDilatation}{#3}{#4}\txb{#1}{#2}{#3}{#4}}
\newcommand{\txcOn}[4]{\txOutBoxOverlay{cbCyan}{\onDilatation}{#3}{#4}\txc{#1}{#2}{#3}{#4}}
\newcommand{\txdOn}[4]{\txOutBoxOverlay{cbGreen}{\onDilatation}{#3}{#4}\txd{#1}{#2}{#3}{#4}}
\newcommand{\txeOn}[4]{\txOutBoxOverlay{cbYellow}{\onDilatation}{#3}{#4}\txe{#1}{#2}{#3}{#4}}
\newcommand{\txfOn}[4]{\txOutBoxOverlay{cbRed}{\onDilatation}{#3}{#4}\txf{#1}{#2}{#3}{#4}}

\newcommand{\expgas}{\ensuremath{\textsf{expected-gas}}\xspace}



\newif\ifisFullPaper
\isFullPaperfalse
% \isFullPapertrue

% Macro to choose between full and short versions
\newcommand{\fullOrShort}[2]{\ifisFullPaper#1\else#2\fi}



% Revision highlighting macros
% \newcommand{\added}[1]{{\textcolor{blue!80!black}{#1}}}
% \newcommand{\removed}[1]{\textcolor{red!80!black}{\sout{#1}}}


\newif\ifaddedhighlight
\addedhighlighttrue 
\ifaddedhighlight
  \newcommand{\added}[1]{{\leavevmode\color{green!40!black}#1}}
\else
  \newcommand{\added}[1]{#1}
\fi

\newif\ifremovedhighlight
\removedhighlighttrue
\ifremovedhighlight
  % \newcommand{\removed}[1]{{\leavevmode\color{red}{#1}}}
  \newcommand{\removed}[1]{}
  \else
  \newcommand{\removed}[1]{#1}
\fi


\newcommand{\mayberemoved}[1]{}
